\documentclass[11pt]{article}
\usepackage[a4paper,left=20mm,right=20mm,top=15mm,bottom=15mm]{geometry}
\usepackage{amsmath}
% \usepackage{amstext}
\usepackage{multicol}
\setlength{\columnsep}{3em}
\setlength{\columnseprule}{0.4pt}
\usepackage{float}
\usepackage{booktabs}
\usepackage{multirow}
\usepackage[UTF8]{ctex}

\title{买卖权平价关系}
\author{杨弘毅}
\date{创建: 2020 年 3 月 20 日 \\修改: \today}
\begin{document}
\maketitle
\section{欧式期权PCP}
使用远期合约(适用于中国市场与美国市场),构建如下两个组合:

\begin{itemize}
\setlength{\itemindent}{2em}
    \item 组合A:欧式看涨多头
    \item 组合B:欧式看跌多头 + 远期合约多头(交割价格(Delivery Price)为K)
\end{itemize}

在T时刻,组合A与组合B价值为:
\begin{multicols}{2}
当$S_T \geq K$时:
\begin{itemize}
\setlength{\itemindent}{2em}
    \item 组合A:$S_T-K$
    \item 组合B:$0 + (S_T-K)$
\end{itemize}
\vfill\columnbreak
当$S_T < K$时:
\begin{itemize}
    \item 组合A:$0$
    \item 组合B:$(K - S_T) + (S_T-K)$
\end{itemize}
\end{multicols}

可以发现组合A与组合B,在T时刻,无论哪种情况下两者价值都相等。因此根据无套利原则,t时刻两个组合价格也必须相同,则有对于远期的PCP公式($S_T=F_T$):
\begin{equation*}
    c = p + (F_T-K)e^{-r(T-t)} \quad\text{或}\quad c = p + S_t - Ke^{-r(T-t)}
\end{equation*}

对于已知红利资产,支付红利现值为$I_t$,红利率为$q$:
\begin{align*}
    c &= p + S_t - I_t - Ke^{-r(T-t)} \quad \text{(已知红利)} \\
    c &= p + S_t e^{-q(T-t)} - Ke^{-r(T-t)}  \quad \text{(已知红利率)} \\
\end{align*}

\section{期权隐含价格}
隐藏期货价格:
\begin{equation*}
    F^*_t = (c-p)e^{r(T-t)} + K
\end{equation*}

隐藏现货价格:
\begin{equation*}
    S^*_t = (c-p) + K e^{r(T-t)}
\end{equation*}

对于中国市场,ETF期权有红利保护。而期货为指数期货无红利保护,自动回落,需要调整红利:
\begin{equation*}
    F^*_t = (c-p)e^{r(T-t)} + K + I_t e^{r(T-t)}
\end{equation*}

\section{美式期权提前行权的影响}
\begin{table}[H]
\centering
\begin{tabular}{@{}cll@{}}
\toprule
\multicolumn{1}{l}{}
& \multicolumn{1}{c}{\textbf{美式看涨期权}} & \multicolumn{1}{c}{\textbf{美式看跌期权}} \\
\midrule
\multirow{2}{*}{\textbf{无红利}} 
& 1.需提前支付行权价,损失行权价格利息 & 1.提前获得行权价格,可获得行前价格利息 \\
& 2.损失期权时间价值 & 2.失去期权时间价值 \\
\textbf{有红利} & 3.提前行权获得股票,因此获得红利 & 3.放弃红利 \\
\bottomrule
\end{tabular}
\end{table}

\section{美式期权PCP}
\subsection{无红利资产}
构造如下两个组合:
\begin{itemize}
\setlength{\itemindent}{2em}
    \item 组合A:欧式看涨多头(c) + K单位现金(用无风险利率投资)
    \item 组合B:美式看跌多头(P,用大写表示) + 一单位股票
\end{itemize}

在\textbf{不提前行权}的情况下,如下所示,组合A的价值都将大于组合B的价值:
\begin{multicols}{2}
当$S_T \geq K$时:
\begin{itemize}
\setlength{\itemindent}{2em}
    \item 组合A:$(S_T-K) + Ke^{r(T-t)}$
    \item 组合B:$S_T$
\end{itemize}
\vfill\columnbreak
当$S_T < K$时:
\begin{itemize}
    \item 组合A:$Ke^{r(T-t)}$
    \item 组合B:$(K - S_T) + S_T$
\end{itemize}
\end{multicols}

只有当$S_\tau<K$时,才会\textbf{提前行权}。此时欧式看涨期权内在价值为零,但仍有时间价值。此时仍有组合A价值大于组合B价值:
\begin{itemize}
\setlength{\itemindent}{2em}
    \item 组合A:$ \text{时间价值} + Ke^{r(T-t)}$
    \item 组合B:$(K-S_\tau) + S_\tau$
\end{itemize}

即无论美式期权是否提前行权,都有组合A价值大于等于组合B的价值,且由于美式期权价格大于等于欧式期权(可提前行权和获得股票,进而获得红利),则有:
\begin{equation*}
    c + K \geq P + S_t \quad \rightarrow \quad C + K \geq P + S_t
\end{equation*}

又由于$P \geq p$(可提前获得行权价格,进而获得行权价的利息),由3欧式期权PCP可得,$P\geq p = c + Ke^{-r(T-t)} - S_t$。当标的资产为无收益资产(无股息红利)时,$c=C$,则有:
\begin{equation*}
    C-P \leq S_t - Ke^{r(T-t)}
\end{equation*}

因此:
\begin{equation*}
   S_t - K \leq C-P \leq S_t - Ke^{r(T-t)}
\end{equation*}

\subsection{有红利资产}
假设股票将派发红利,其现值$I_t$,构造如下两个组合,注意组合B由于持有股票,将在T时刻获得红利$I_t e^{r(T-t)}$:
\begin{itemize}
\setlength{\itemindent}{2em}
    \item 组合A:欧式看涨多头(c) + (K+I)单位现金(用无风险利率投资)
    \item 组合B:美式看跌多头(P,用大写表示) + 一单位股票
\end{itemize}

在\textbf{不提前行权}的情况下,组合A的价值都将大于组合B的价值:
\begin{multicols}{2}
当$S_T \geq K$时:
\begin{itemize}
\setlength{\itemindent}{2em}
    \item 组合A:$(S_T-K) + (K+I_t)e^{r(T-t)}$
    \item 组合B:$S_T + I_t e^{r(T-t)}$
\end{itemize}
\vfill\columnbreak
当$S_T < K$时:
\begin{itemize}
    \item 组合A:$(K+I_t)e^{r(T-t)}$
    \item 组合B:$K- S_T + S_T + I_t e^{r(T-t)}$
\end{itemize}
\end{multicols}

同理,只有在$S_\tau<K$时,才会\textbf{提前行权},假设此时时刻为$t<\tau<T$,虽然看涨期权内在价值为0,但仍有时间价值,则有:
\begin{itemize}
\setlength{\itemindent}{2em}
    \item 组合A:$ 时间价值 + (K+I_t)e^{r(T-t)}$
    \item 组合B:$(K-S_\tau) + S_\tau + I_t e^{r(T-t)}$
\end{itemize}

即无论是否提前行权,都有组合A价值大于组合B价值,同时因$C \geq c$,则有:
\begin{equation*}
    c + K + I \geq P + S_t \quad \rightarrow \quad C + K +I \geq P + S_t
\end{equation*}

即:
\begin{equation*}
    C-P \geq S_t - K - I
\end{equation*}

因有红利资产,再发放红利之后股价自然回落,因此看涨期权价格更低,看跌期权更高,$C-P$组合价值更低,并由无红利资产美式期权推导得:
\begin{equation*}
    C-P \leq S_t - Ke^{r(T-t)}
\end{equation*}

因此:
\begin{equation*}
   S_t - K - I \leq C-P \leq S_t - Ke^{r(T-t)}
\end{equation*}


\section{中国ETF期权PCP}
ETF期权受红利保护,则有PCP为:
\begin{equation*}
    c + Ke^{-r(T-t)} = p + F_t e^{-r(T-t)} + I_t 
\end{equation*}

\end{document}