\documentclass[11pt]{article}
\usepackage{amsmath}
% for underline word wrap
\usepackage{ulem}
\usepackage[a4paper,left=20mm,right=20mm,top=15mm,bottom=15mm]{geometry}
\usepackage{tikz}
\usetikzlibrary{trees,matrix}
\usepackage[UTF8]{ctex}
% ctex与amsmath冲突,放最后可解决
\title{三种定价衍生品定价方式}
\author{杨弘毅}
\date{创建: 2020 年 2 月 26 日 \\修改: \today}
\begin{document}
\maketitle

\section{衍生品定价基本假设}
衍生品定价一般使用相对定价法,即利用标的资产价格与衍生品价格之间的内赞关系,根据标的资产的价格求出衍生品价格。与绝对定价法不同,绝对定价法为使用适当的贴现率将未来现金流贴现加总,但贴现率和未来现金流都难以预测。相对定价法则有贴近市场,易于实现的优点,以下为衍生品定价基本假设:
\begin{itemize}
  \item 市场不存在摩擦
  \item 市场是完全竞争
  \item 市场不存在无风险套利机会
  \item 市场不存在对手风险
\end{itemize}

\section{简单例子}
市价为10元,三个月后,该股票价格存在两种状态,或上涨为11元,或下跌至9元。假设无风险年利率$r_f=10\%$,为三个月协议价格为10.5元的以该股票为标的资产的看涨期权定价。
\begin{figure}[!ht]
  \centering
  \begin{tikzpicture}[>=stealth,sloped]
    \matrix (tree) [%
      matrix of nodes,
      minimum size=1cm,
      column sep=1.5cm,
      row sep=0.6cm,
    ]
    {
               & { $S_1=11$, $c_1=0.5$} \\
      $S_0=10$ &                        \\
               & { $S_1=9$, $c_1=0$}    \\
    };
    \draw[->] (tree-2-1) -- (tree-1-2) node [midway,above] {$p$};
    \draw[->] (tree-2-1) -- (tree-3-2) node [midway,below] {$1-p$};
  \end{tikzpicture}
  \caption{两种资产,两种状态} \label{fig:M1}
\end{figure}

\subsection{复制定价}
第一种方法,使用股票和期权构建无风险债券组合。即由$\Delta$单位股票多头和一单位看涨空头构建,无论上涨下跌状态下,在T时刻,其价值都应等于K单位无风险债券(零息债券,Zero-coupon Bond,$B_T=1$)。上涨状态下$K=11\Delta-0.5$,下跌状态下$K=9\Delta$,得到$11\Delta-0.5=9\Delta=K$,即$\Delta=0.25$。即有使用0.25单位的股票多头和一单位的看涨期权空头,即可构建K单位无风险资产$K=2.25$。在t时刻,有$0.25S_0 - c_0=2.25e^{-r_f(T-t)}$,有$c_0=0.30555$。

第二种方法,使用无风险资产和股票复制期权。假设买进$\Delta$单位股票,K单位无风险资产,在T时刻有:
\begin{equation*}
  \left\{
  \begin{aligned}
    11\Delta + K  &= 0.5 \\
    9\Delta + K  &= 0 
  \end{aligned}
  \right.
  \qquad
  \Rightarrow
  \qquad
  \left\{
  \begin{aligned}
  \Delta &= 0.25 \\
  K &= -2.25 
  \end{aligned}
  \right.
\end{equation*}

即可以使用0.25单位股票多头,以及2.25单位无风险债券空头复制期权。在t时刻,$c_0=0.25S_0-2.25e^{r_f(T-t)}=0.30555$。

\subsection{状态价格定价}
使用状态价格证券(State-price Security)为衍生品定价。假设股票价格上涨状态回报为1(T时刻)的状态价格证券在t时刻的价格,称为状态价格$PC_u$。同样,股票价格下跌状态回报为1(T时刻)的状态价格(t时刻)为$PC_d$。
\begin{equation*}
PC_u = 
\left\{
\begin{aligned}
1, \quad \text{上涨} \\
0, \quad \text{下跌} 
\end{aligned}\right.
\hspace*{8em}
PC_d = 
\left\{
\begin{aligned}
0, \quad \text{上涨} \\
1, \quad \text{下跌} 
\end{aligned}\right.
\end{equation*}

使用状态价格证券复制出股票收益,以及无风险债券收益。
\begin{equation*}
\left\{
\begin{aligned}
11PC_u + 9PC_d = 10, \quad \text{股票} \\
11PC_u + 9PC_d = 10, \quad \text{债券} 
\end{aligned}\right.
\qquad
\Rightarrow
\qquad
\left\{
\begin{aligned}
PC_u = 0.611105 \\
PC_d = 0.364205 
\end{aligned}\right.
\end{equation*}

由期权收益可知$c_0=0.5PC_u \times 0PC_d=0.30555$。

\subsection{风险中性定价}
使用风险中性定价法,假设风险中性下股票上涨概率为$Q$,股票下跌的概率为$1-Q$。根据资产定价基本原理,$P_t = E_t[M_{t+1}X_{t+1}]$,其中$M_t$又被称为随机贴现因子(Stochastic discount factor,SDF)或称为定价核(Pricing kernel),(贴现率$M_{t+1}=e^{-r_f(T-t)}$已知)。
\begin{equation*}
 S_0 = e^{-R_f(T-t)}[11Q+9(1-Q)] = 10
 \qquad
 \Rightarrow
 \qquad
 Q = 0.626576
\end{equation*}

此时可计算出期权价格:
\begin{equation*}
 c_0 = M^Q[Q_1X_1 + Q_2X_2] = e^{-R_f(T-t)}[0.5Q+0(1-Q)] = 0.30555
\end{equation*}

\section{复杂例子}
在t时刻,市场上有两种互相不可复制的可交易资产$S_a=8$元和$S_b=9$元,且未来T时刻有三种状态,其回报分别问(10,8,0)和(8,12,0)。求行权价为7元,标的资产为$S_a$的看涨期权价格$C_a$。(注意:没有无风险债券,只有资产$S_a$和$S_b$)
\begin{figure}[!ht]
  \centering
  \begin{tikzpicture}[>=stealth,sloped]
    \matrix (tree) [%
      matrix of nodes,
      minimum size=1cm,
      column sep=1.5cm,
      row sep=0.6cm,
    ]
    {
               &  10 \\
      $S_a=8$  &  8  \\
               &  0  \\
    };
    \draw[->] (tree-2-1) -- (tree-1-2) node [midway,above] {};
    \draw[->] (tree-2-1) -- (tree-2-2) node [midway,above] {};
    \draw[->] (tree-2-1) -- (tree-3-2) node [midway,below] {};
  \end{tikzpicture}
  \qquad
  \begin{tikzpicture}[>=stealth,sloped]
    \matrix (tree) [%
      matrix of nodes,
      minimum size=1cm,
      column sep=1.5cm,
      row sep=0.6cm,
    ]
    {
               &  8  \\
      $S_b=9$  &  12  \\
               &  0  \\
    };
    \draw[->] (tree-2-1) -- (tree-1-2) node [midway,above] {};
    \draw[->] (tree-2-1) -- (tree-2-2) node [midway,above] {};
    \draw[->] (tree-2-1) -- (tree-3-2) node [midway,below] {};
  \end{tikzpicture}
  \qquad
  \begin{tikzpicture}[>=stealth,sloped]
    \matrix (tree) [%
      matrix of nodes,
      minimum size=1cm,
      column sep=1.5cm,
      row sep=0.6cm,
    ]
    {
               &  3  \\
      $c_a=10$ &  1  \\
               &  0  \\
    };
    \draw[->] (tree-2-1) -- (tree-1-2) node [midway,above] {};
    \draw[->] (tree-2-1) -- (tree-2-2) node [midway,above] {};
    \draw[->] (tree-2-1) -- (tree-3-2) node [midway,below] {};
  \end{tikzpicture}
  \caption{三种资产,三种状态} \label{fig:M2}
\end{figure}

\subsection{复制定价}
使用$S_a$和$S_b$复制看涨期权回报,a和b分别代表复制期权所需$S_a$以及$S_b$的资产数量
\begin{equation*}
  \left\{
    \begin{aligned}
    10a+8b&=3 \\
    8a+12b&=1 \\
    0a+0b&=0
    \end{aligned}
  \right.
  \qquad
  \Rightarrow
  \qquad
  \left\{
    \begin{aligned}
    a&=0.5 \\
    b&=-0.25
    \end{aligned}\right.
\end{equation*}

得到$a=0.5 b=-0.25$,即使用0.5份$S_a$资产多头和0.25份$S_b$资产空头,即可复制期权。从而得到在t时刻,期权价格为$c_a=0.5S_a-0.25S_b=1.75$元。

\subsection{状态价格定价}

其定价思路为先用可交易资产计算出状态价格证券价格$PC_1,PC_2,\ldots,PC_n$,而后再用状态价格证券为其他证券定价。假设三种状态价格证券,分别为$PC_1$,$PC_2$和$PC_3$,其回报为(1,0,0),(0,1,0)以及(0,0,1)。如下所示:
\begin{equation*}
PC_1 = 
\left\{
\begin{aligned}
1, \quad \text{状态1} \\
0, \quad \text{状态2} \\
0, \quad \text{状态3} 
\end{aligned}\right.
\qquad
PC_2 = 
\left\{
\begin{aligned}
0, \quad \text{状态1} \\
1, \quad \text{状态2} \\
0, \quad \text{状态3} 
\end{aligned}\right.
\qquad
PC_3 = 
\left\{
\begin{aligned}
0, \quad \text{状态1} \\
0, \quad \text{状态2} \\
1, \quad \text{状态3} 
\end{aligned}\right.
\end{equation*}

由此可复制$S_a$和$S_b$回报,并计算期权价格$c_a=3 PC_{1} + 1 PC_2 =1.75$元。
\begin{equation*}
  \left\{
    \begin{aligned}
    10PC_1+8PC_2+0PC_3&=8 \\
    8PC_1+12PC_2+0PC_3&=9
    \end{aligned}
  \right.
  \qquad
  \Rightarrow
  \qquad
  \left\{
    \begin{aligned}
    PC_1 = 0.42857\\
    PC_2 = 0.46429
    \end{aligned}
  \right.
\end{equation*}

状态价格为未来特定状态下的1元的现值。所以状态价格应取决于三个因素:状态发生概率、对不同状态的风险偏好(不同状态之间现值的差异)以及无风险利率(贴现率)。

\subsection{风险中性定价}
假设状态1对应概率为$P_1$,状态2对应概率为$P_2$,对应状态3的概率为$1-P_1-P_2$。同样根据资产定价原理$P_t=E_t[M_{t+1}X_{t+1}]$,则有$S=P_1M_1X_1 + P_2M_2X_2+(1-P_1-P_2)M_3X_3$。在状态价格定价中,有$\Pi_i = P_iM_i$使得方程可以求解。但使用风险中性定价,$M_1$、$M_2$、$M_3$、$P_1$、$P_2$和$C_a$均为未知,无法求解。即使市场为完全市场(三种资产和三种状态),但三种资产即有三个方程,六个未知数,方程组存在无穷多解(现实世界中状态数远大于资产数)。\textbf{注意:1.$M_i$为随机贴现因子只取决于状态,即对状态的风险态度,而不随着资产不同而改变。2.$E_t[MX]$只有当其为无风险贴现因子时可写成$ME_t[X]$,因为无风险利率$r_f$为常数且不因状态改变而改变。}
\begin{equation*}
  \left\{
    \begin{aligned}
      S_a &= P_1M_1 \times 10 + P_2M_2 \times 8 + (1-P_1-P_2)M_3 \times 0 = 8\\
      S_b &= P_1M_1 \times 8 + P_2M_2 \times 12 + (1-P_1-P_2)M_3 \times 0 = 9\\
      C_a &= P_1M_1 \times 3 + P_2M_2 \times 1 + (1-P_1-P_2)M_3 \times 0 \\
    \end{aligned}
  \right.
\end{equation*}

我们可以对多余变量自由设定,并不影响等式成立,\uline{最简单的设定就是令三种状态下贴现因子都相等,都为$M^Q$无风险贴现因子}。此时投资者对风险持中性态度,即对未来不存在任何偏好。由于对不同的状态下贴现因子都相同,可把$M^Q$提出为$M^QE_t[X]$。由$Q_i M^Q = P_i M_i$(即$PC_i$),得到$Q_i = P_i \frac{M_i}{M_Q}$,则此时有$S = M^Q E^Q[X] = M^Q[Q_1X_1 + Q_2X_2 + Q_3X_3]$,简化方程组得:
\begin{equation*}
  \left\{
    \begin{aligned}
      S_a &= P_1M_1 \times 10 + P_2M_2 \times 8  &= 8 \\
      S_b &= P_1M_1 \times 8 + P_2M_2 \times 12  &= 9
    \end{aligned}
  \right.
  \qquad
  \Rightarrow
  \qquad
  \left\{
    \begin{aligned}
    P_1M_1 &= 3/7 \\
    P_2M_2 &= 13/28
    \end{aligned}
  \right.
\end{equation*}

在风险中性下:
\begin{equation*}
  \left\{
  \begin{aligned}
    Q_1 &= P_1M_1 / M^Q = \frac{3}{7 M^Q} \\ 
    Q_2 &= P_2M_2 / M^Q = \frac{13}{28 M^Q}\\
    Q_3 &= 1 - \frac{25}{28 M^Q}
  \end{aligned}
  \right.
\end{equation*}

计算期权价格:
\begin{equation*}
 c_a = M^Q E^Q[X] = M^Q [3Q_1 + Q_2 + 0Q_3] = M^Q [3\frac{3}{7M^Q} + \frac{13}{28M^Q}] = 1.75
\end{equation*}

所以风险中性定价法基本思路为,为了方便定价过程,我们可以从现实测度转换至风险中性测度,而定价结果同样适用于现实测度。即将现实测度P转化为Q测度,进行测度转换,在Q测度下计算预期回报$E^Q[X]$,并且使用无风险贴现因子$M^Q$贴现为现值。

\section{概念}

\subsection*{可复制证券}
在这个例子中只有两种可交易资产,有三种状态,为不完全市场(Imcomplete Market)。但由于需要定价的期权,在$S_0$,$S_1$生成的平面上,可以被复制,因此可以被准确定价。而对于\textbf{不可复制资产},虽然无法准确定价,但可以确定$M^Q$的合理范围,从确定新资产的价格合理范围。\textbf{(注意:可复制不代表是完全市场)}

\subsection*{不可复制证券}
假设需要为t=1时刻回报为(3,2,1)的资产$S_3$进行定价,因为其不可被复制,只能确定其范围。当$Q_3=0$时,$Q_1+Q_2\leq1$,由$M^Q(Q_1 + Q_2) = 25/28$得到$M^Q\geq25/28$。因为随机贴现因子$M^Q\leq1$,确定其范围为$ 25/28\leq M^Q\leq1$。当$M^Q=25/28$时,$Q_1=12/25$,$Q_2=13/25$,此时资产价格$S_3=2.21$。当$M^Q=1$时,$Q_1=3/7$,$Q_2=13/28$,$Q_3=3/28$,此时资产价格$S_3 = 2.32$。得到该资产$S_3$价格的合理范围在$2.21 \sim 2.32$之间。

\subsection*{完全市场}
在完全市场中(Complete Market),三种资产($S_a$、$S_b$和$S_c$)对应三种状态,在风险中性测度下:
\begin{equation*}
  \left\{
    \begin{aligned}
      S_a &= M^Q[Q_1X_{a,1} + Q_2X_{a,2} + (1-Q_1-Q_2)X_{a,3}] \\
      S_b &= M^Q[Q_1X_{b,1} + Q_2X_{b,2} + (1-Q_1-Q_2)X_{b,3}] \\
      S_c &= M^Q[Q_1X_{c,1} + Q_2X_{c,2} + (1-Q_1-Q_2)X_{c,3}]
    \end{aligned}
  \right.
\end{equation*}

如果$(X_{a,1},X_{a,2},X_{a,3})$,$(X_{b,1},X_{b,2},X_{b,3})$和$(X_{c,1},X_{c,2},X_{c,3})$为线性独立(相互不可复制),即能生成整个三维线性空间,即\textbf{空间内任意资产都可复制}。此时有三个未知数$Q_1$,$Q_2$和$M^Q$,三个方程,能求解所有未知数。相反,在现实测度下则无法求解所有未知数,只能整体求解$PC_i = P_iM_i$,即求解状态价格证券。如上述例子中,资产$S_a$与$S_b$之间互相不可复制,即互相独立,但可由这两种证券复制出第三种证券$c_a$,市场为可复制,但非完美市场。即$c_a$在$S_a$与$S_b$生成的平面上,但$S_a$和$S_b$无法生成整个三维的状态空间,并非任何资产都可被复制。\textbf{(注意:在可复制基础上,需要无套利条件,资产才能被定价)}

\begin{equation*}
  \left\{
    \begin{aligned}
      S_a &= P_1M_1X_{a,1} + P_2M_2X_{a,2} + P_3M_3X_{a,3} \\
      S_b &= P_1M_1X_{b,1} + P_2M_2X_{b,2} + P_3M_3X_{b,3} \\
      S_c &= P_1M_1X_{c,1} + P_2M_2X_{c,2} + P_3M_3X_{c,3}
    \end{aligned}
  \right.
\end{equation*}

\subsection*{完美市场}
完美市场(Perfect Market),指没有交易摩擦的市场,能够自由做空做多。

\end{document}
