\documentclass[11pt]{article}
\usepackage{amsmath}
% \usepackage{amstext}
\usepackage{amsfonts}
\usepackage{booktabs}
\usepackage{float}
\usepackage{multirow}
\usepackage[a4paper,left=10mm,right=10mm,top=15mm,bottom=15mm]{geometry}
\usepackage[UTF8]{ctex}

\newcommand{\E}{\operatorname{E}}
\newcommand{\Cov}{\operatorname{Cov}}
\newcommand{\Var}{\operatorname{Var}}

% \def\Var{{\textrm{Var}}\,}
% \def\Cov{{\textrm{Cov}}\,}
% \def\E{{\textrm{E}}\,}

\title{BSM公式}
\author{杨弘毅}
\date{创建: 2020 年 4 月 19 日 \\修改: \today}

\begin{document}
\maketitle

\section{布朗运动、维纳过程}

标准布朗运动简易表达式有:
\begin{equation*}
    dZ_t = \varepsilon_t \sqrt{dt}
\end{equation*}

其离散形式的表达式有:
\begin{equation*}
    Z_T - Z_t = \sum^n_{i=1} \varepsilon_i \sqrt{\Delta t}
\end{equation*}

\subsection{特征}
\textbf{标准布朗运动(Brownian motion)或维纳过程(Wiener process)}的特征有:
\begin{itemize}
    \item 初值为零
    \item 连续
    \item 独立增量:对于任意两个不同时间点$\Delta t_i$与$\Delta t_j$,其增量$\Delta Z_i$与$\Delta Z_j$相互独立
    \item 独立同分布(方差可加):增量$\Delta Z$服从均值为零、方差等于时间长度的正态分布,即$\Delta Z_i  \sim N(0,\Delta t_i)$
\end{itemize}

\subsection{为何使用标准布朗运动}

\begin{itemize}
    \item 股价不能为负,所以不能遵循正态分布,但股票连续复利收益率($d\text{ln} S_t$)近似服从正态分布
    \item 维纳过程是一个马尔可夫随机过程,增量$\Delta Z$独立,与弱式EMH相同,即技术分析无效,无法使用历史信息预测未来,过去信息跟未来信息相互独立
    \item 维纳过程对时间处处不可导,且二次变分(Quadratic Variation)不为零,与股票价格变化存在转折尖点的性质相符
\end{itemize}

\subsection{部分证明}

\textbf{增量均值为零,方差为时间长度},当$X$与$Y$独立时,则有:
\begin{equation*}
    \Var(XY) = \Var(X)\Var(Y) + [\E(X)]^2 \Var(Y) + [E(Y)]^2 \Var(X)
\end{equation*}

此时,由于$\varepsilon_t$与$dt$独立,套用上式,同时由于$\varepsilon_t \sim N(0,1)$,则有:
\begin{equation*}
    \E(dZ_t) = \E(\varepsilon_t \sqrt{dt}) = 0
\end{equation*}

\begin{align*}
    \Var(dZ_t) & = \Var( \varepsilon_t \sqrt{dt}) \\
    & =  \Var(\varepsilon_t) \Var(\sqrt{dt}) + [\E(\varepsilon_t)]^2 \Var(\sqrt{dt}) + [E(\sqrt{dt})]^2 \Var(\varepsilon_t) \\
    & = \Var(\varepsilon_t) \left[ \Var[(\sqrt{dt})^2] - [\E(\sqrt{dt})]^2 \right] \\
    & = \Var(\varepsilon_t) \left[ \E[(\sqrt{dt})^2] - [\E(\sqrt{dt})]^2 + [\E(\sqrt{dt})]^2 \right] \\
    & = 1 \cdot \E(dt) = dt
\end{align*}

\textbf{方差可加性},由下式可见,由于独立增量,导致协方差项为零,使得方差可加。
\begin{align*}
     & \Var(X_1+X_2+X_3) \\
     & = \Var(X_1) + \Var(X_2) + \Var(X_3) \\
     & + \Cov(X_1,X_2) + \Cov(X_2,X_3) + \Cov(X_1,X_3)
\end{align*}

由上可知,增量在连续形式$dZ_t$以及离散形式$Z_T-Z_t$下,均服从均值为零,方差为时间长度的正态分布,即有:
\begin{align*}
    dZ_t & \sim N(0,dt) \\
    Z_T - Z_t & \sim N(0,T-t)
\end{align*}

\subsection{几种随机过程}

\textbf{广义维纳过程(generalized Wiener process)},a与b为常数。此时,易知其均值为$\E(dX_t) = adt$,由于$b$为常数,且$\Var(dZ_t)=dt$,则有方差为$\Var(dX_t)=b^2 dt$。
\begin{equation*}
    dX_t =a dt + b dZ_t
\end{equation*}

\textbf{普通布朗运动},a(t)与b(t)都是t的确定性函数。由于都为确定函数,所以如上可知,其均值方差为$\E(dX_t) = a(t)dt$,由于$b$为常数,且$\Var(dZ_t)=dt$,则有方差为$\Var(dX_t)=b(t)^2 dt$。
\begin{equation*}
    dX_t =a(t) dt + b(t) dZ_t
\end{equation*}

\textbf{扩散过程(Diffusion Process)},此时$a(X(t),t)$与$b(X(t),t)$都为$X_t$和$t$的确定性函数。由于漂移项与方差项都包含$X(t)$,使得扩散之后过程的条件分布无法保证仍是正态分布。但更能刻画一般动态变化,未加入新的风险源,仍具有独立增量,马尔可夫性,和方差可加性等性质。
\begin{equation*}
    dX_t =a(X(t),t) dt + b(X(t),t) dZ_t
\end{equation*}

\textbf{伊藤过程(Itô Process)},最一般化的随机过程,$a_t$和$b_t$为任意函数或随机过程。
\begin{equation*}
    dX_t =a_t dt + b_t dZ_t
\end{equation*}

\section{伊藤引理(Itô lemma)}

若变量$X_t$遵循伊藤过程:
\begin{equation*}
    dX_t =a_t dt + b_t dZ_t
\end{equation*}

在导数$\partial G/\partial t$、$\partial G/\partial X$与$\partial^2 G/\partial X^2$存在的前提下,则有变量$X_t$和$t$的函数$G(X_t,t)$将遵循如下过程:
\begin{equation*}
    dG_t = \left(\frac{\partial G}{\partial X}a_t  + \frac{\partial G}{\partial t} + \frac{1}{2}\frac{\partial^2 G}{\partial X^2} b^2_t \right)dt + \frac{\partial G}{\partial X} b_t dZ_t
\end{equation*}

\subsection{证明}

$G(X,t)$的泰勒展开式为:
\begin{equation*}
    \Delta G_t = \frac{\partial G}{\partial X} \Delta X + \frac{\partial G}{\partial t} \Delta t + \frac{1}{2} \frac{\partial^2 G}{\partial X^2}\Delta X^2  + \frac{\partial G}{\partial X \partial t}\Delta X \Delta t + \frac{1}{2} \frac{\partial^2 G}{\partial t^2} \Delta t^2 + \dots
\end{equation*}

当$\Delta t \rightarrow 0$时,$(\Delta t)^2$,认为是高阶无穷小,可忽略。而对于$\Delta X \Delta t$项有:

\begin{align*}
    \Delta X &= a\Delta t + b \varepsilon\sqrt{\Delta t} \\
    \Delta X \Delta t &= a(\Delta t)^2 + b \varepsilon(\Delta t)^{3/2}
\end{align*}

其中的$(\Delta t)^{3/2}$项,也被认为时高阶无穷小项,可忽略。同时由于$(\Delta X)^2$项中包含$\Delta t$项,因此需要保留。因此仅考虑前三项(\textbf{注意}:此与常微分不同,而在常微分中,$(\Delta X)^2$项是也是高阶无穷小项),展开得到:
\begin{align*}
    \Delta G_t & = \frac{\partial G}{\partial X} \Delta X + \frac{\partial G}{\partial t} \Delta t + \frac{1}{2} \frac{\partial^2 G}{\partial X^2}\Delta X^2 \\
    & = \frac{\partial G}{\partial X} \Delta X + \frac{\partial G}{\partial t} \Delta t + \frac{1}{2} \frac{\partial^2 G}{\partial X^2} [a\Delta t + b\varepsilon\sqrt{\Delta t}]^2 \\
    & = \frac{\partial G}{\partial X} \Delta X + \frac{\partial G}{\partial t} \Delta t + \frac{1}{2} \frac{\partial^2 G}{\partial X^2} b^2 \varepsilon^2 \Delta t
\end{align*}

对于$\varepsilon^2 \Delta t$项,由于$\varepsilon \sim N(0,1)$,因此有$\E(\varepsilon)=0$。又因$\Var(\varepsilon)=\E(\varepsilon^2)-[\E(\varepsilon)]^2=1$,得到$\E(\varepsilon^2)=1$,同时有$\E(\varepsilon^2 \Delta t) = \Delta t$。计算$\varepsilon^2 \Delta t$的方差可得:
\begin{align*}
    \Var(\varepsilon^2 \Delta t) & = \Var(\varepsilon^2) \Var(\Delta t) + [\E(\varepsilon^2)]^2 \Var(\Delta t) + [E(\Delta t)]^2 \Var(\varepsilon^2) \\
    & = \Var(\varepsilon^2) \Var(\Delta t) + 1 \cdot \Var(\Delta t) + [E(\Delta t)]^2 \Var(\varepsilon_t^2) \\
    & = \mathcal{O}(\Delta t^2)
\end{align*}

可以认为$\varepsilon^2 \Delta t$方差为高阶无穷小,其期望为$1$。因此,可认为$\varepsilon^2 \Delta t \approx \Delta t$,可将原式化简为:
\begin{equation*}
    \Delta G_t = \frac{\partial G}{\partial X} \Delta X + \frac{\partial G}{\partial t} \Delta t + \frac{1}{2} \frac{\partial^2 G}{\partial X^2} b^2 \Delta t
\end{equation*}

而连续形式为:
\begin{align*}
    dG_t & = \frac{\partial G}{\partial X}  dX_t + \frac{\partial G}{\partial t} dt + \frac{1}{2} \frac{\partial^2 G}{\partial X^2} b^2 dt \\
    & = \frac{\partial G}{\partial X} (a_t dt + b_t dZ_t) + \frac{\partial G}{\partial t} dt + \frac{1}{2} \frac{\partial^2 G}{\partial X^2} b^2 dt \\
    & = \left(\frac{\partial G}{\partial X}a_t  + \frac{\partial G}{\partial t} + \frac{1}{2}\frac{\partial^2 G}{\partial X^2} b^2_t \right)dt + \frac{\partial G}{\partial X} b_t dZ_t
\end{align*}


\section{几何布朗运动}

由于衍生品价格是标的资产价格与时间的函数,即只需要假定标的资产遵循过程,即可用伊藤引理求得其衍生品遵循过程。假设股票价格服从几何布朗运动(Geometric Brownian Motion,GBM):
\begin{equation*}
    dS_t = \mu S_t d_t + \sigma S_t dZ_t
\end{equation*}

令$G_t = \ln S_t$,此时:
\begin{equation*}
    \frac{\partial G}{\partial S} = \frac{1}{S_t}, \quad
    \frac{\partial^2 G}{\partial S^2} = -\frac{1}{S_t^2}, \quad
    \frac{\partial G}{\partial t} = 0
\end{equation*}

代入伊藤引理之中,此时$a_t=\mu S_t$,$b_t=\sigma S_t$,则有:
\begin{align*}
    dG_t = d \ln S_t & = \left( \frac{1}{S_t}\mu S_t + 0 - \frac{1}{2} \frac{1}{S_t^2} \sigma^2 S_t^2 \right) dt + \frac{1}{S_t}\sigma S_t dZ_t \\
    & = \left( \mu - \frac{1}{2}\sigma^2\right)dt + \sigma dZ_t
\end{align*}

即有:
\begin{align*}
    d\ln S_t= \left( \mu - \frac{1}{2}\sigma^2\right) dt + \sigma dZ_t \sim N \left( (\mu-\frac{\sigma^2}{2})dt, \sigma^2 dt \right)
\end{align*}

同时又离散形式下:
\begin{align*}
    \ln S_T - \ln S_t = \left( \mu - \frac{\sigma^2}{2} \right) \Delta t + \sigma (Z_T - Z_t) \sim N \left((\mu-\frac{\sigma^2}{2})(T-t), \sigma^2(T-t) \right)
\end{align*}

此时股票价格连续复利收益率(Continuously compounded return),或称为对数收益率(Logarithmic return),为\textbf{未年化}收益率:
\begin{equation}
    R = d\ln S_t= \ln S_t - \ln S_{t-1} = \ln \frac{S_t}{S_{t-1}} = \ln (1+r)
\end{equation}

服从期望值为$(\mu - \sigma^2/2)dt$,方差为$\sigma^2 dt$的\textbf{正态分布},与现实较为吻合。且$d\ln S_t$的定义,使得股票价格非负。\textbf{注意}:$d\ln S$(极短时间内)和$\ln S_T - \ln S_t$(较长时间内)都服从正态分布,而$dS$在极短时间内服从正态分布,而在较长时间内因$S_t$的大小改变,使得$S_T - S_t$的均值和方差的改变而不服从正态分布。
\begin{align*}
    \ln S_T - ln S_t \sim & N \left( (\mu-\frac{\sigma^2}{2})(T-t), \sigma^2(T-t)\right) \\
    \ln S_T  \sim & N \left(\ln S_t + (\mu-\frac{\sigma^2}{2})(T-t), \sigma^2(T-t)\right)
\end{align*}

\textbf{可以看到此时\underline{股票价格的对数}服从正态分布(Log-normal distribution),因此可知,股票价格服从对数正态分布}。由正态分布与对数正态分布的性质可知,对一个服从正态分布的随机变量$X$取指数,则$e^X$服从数正态分布。相反,对一个服从对数正态分布的随机变量$X$取对数,则$\ln X$服从正态分布。因此有如下关系:
\begin{align*}
    \ln S_T \sim N(\mu,\sigma^2) \quad \leftrightarrow \quad S_T \sim \text{Log-normal}(\mu,\sigma^2)
\end{align*}

并且已知对数正态分布$X \sim \text{Log-normal}(\mu,\sigma^2)$,其期望与标准差为:
\begin{align*}
    \E(X) & = e^{\mu+\frac{\sigma^2}{2}} \\
    \Var(X) & = e^{2\mu+\sigma^2} (e^{\sigma^2}-1)
\end{align*}

因股票价格$S_T$服从对数正态分布,代入上式可知其期望及方差为:
\begin{align*}
    \E(S_T)   & = \exp(\ln S_t + (\mu - \frac{\sigma^2}{2})(T-t)+\frac{\sigma^2}{2}(T-t)) \\
    & = \exp(\ln S_t + \mu(T-t)) \\
    & = S_t e^{\mu(T-t)} \\
    \Var(S_T) & = \left[\exp(\sigma^2(T-t))-1\right] \exp \left\{2\left[\ln S_t + (\mu - \frac{\sigma^2}{2})(T-t)\right] + \sigma^2(T-t)\right\} \\
    & = \left[\exp(\sigma^2(T-t))-1\right] \exp\left[2 \ln S_t + 2\mu (T-t)\right] \\
    & = S_t^2 e^{2\mu(T-t)} \left[ e^{\sigma^2 (T-t)} - 1 \right]
\end{align*}

\subsection{对数正态分布}

如果一组数值做对数变换后服从正态分布,我们就称其服从对数正态分布。假设随机变量$X$服从正态分布,则有$ln x$服从对数正态分布,两者累积分布函数(Cumulative distribution fuction,CDF)相同:
\begin{equation*}
    F_L(x) = F_N (\ln x)
\end{equation*}

对公式两边取倒数,则可得到其概率密度函数(probability density function,PDF):
\begin{equation*}
    F_L(x) = \frac{1}{x} F_N(\ln x)
\end{equation*}

此时,带入已知正态分布PDF,即可得到对数正态分布PDF:
\begin{equation*}
    f_L = \frac{1}{x\sigma\sqrt{2\pi}} e^{-\frac{(\ln x - \mu)^2}{2\sigma^2}}
\end{equation*}

\textbf{注意}:对于对数正态分布,$\mu$与$\sigma$并非其均值与标准差,仅为确定其对数正态分布的两个参数。只是使用其确定正态分布时,就正好为其期望和标准差。对于相同的$\mu$与$\sigma$参数确定的正态分布与对数正态分布,可以通过对服从对数正态分布的随机变量取对数转换为正态分布。相反,通过对服从正态分布的随机变量取指数转换为对数正态分布。两者之间的期望与标准差(方差)通过如下关系转化:

\begin{table}[H]
\centering
\begin{tabular}{@{}cll@{}}
\toprule
\multicolumn{1}{l}{}
& \multicolumn{1}{c}{\textbf{正态分布}} & \multicolumn{1}{c}{\textbf{对数正态分布}} \\
\midrule
\multirow{1}{*}{\textbf{期望}} 
& $\E_N(X)=\mu = \ln[\E_L(X)] - \frac{1}{2} \ln \left[1+\frac{\Var_L(X)}{[\E_L(X)]^2}\right] $ & $\E_L(X) = e^{\mu+\frac{\sigma^2}{2}}$ \\
\textbf{方差} & $\Var_N(X) = \sigma^2 = \ln \left[1+\frac{\Var_L(x)}{[\E_L(X)]^2}\right]$ & $\Var_L(X) = e^{2\mu+\sigma^2} 
\left(e^{\sigma^2} - 1\right)$ \\
\bottomrule
\end{tabular}
\end{table}

\subsubsection{期望推导}

根据对数正态分布的PDF,可计算其期望:
\begin{equation*}
    \E(X) = \int^{+\infty}_0 x f(x) dx = \int^{+\infty}_0 \frac{1}{\sigma\sqrt{2\pi}} e^{-\frac{(ln x - \mu)^2}{2\sigma^2}} dx
\end{equation*}

使用换元法,令$t = \frac{ln x - \mu}{\sqrt{2}\sigma}$,则有$x = e^{\sqrt{2}\sigma t + \mu}$,则原积分转化为:
\begin{align*}
    \E(X) &= \int^{+\infty}_{-\infty} \frac{1}{\sigma\sqrt{2\pi}}e^{-t^2}d e^{\sqrt{2}\sigma t + \mu}\\ 
    &= \frac{e^{\mu + \frac{\sigma^2}{2}}}{\sqrt{\pi}} \int^{+\infty}_{-\infty} e^{-(t-\frac{\sqrt{2}\sigma}{2})^2} dt\\ 
    &= \frac{e^{\mu + \frac{\sigma^2}{2}}}{\sqrt{\pi}} \int^{+\infty}_{-\infty} e^{-t^2}d e^{-(t-\frac{\sqrt{2}\sigma}{2})^2} d(t-\frac{\sqrt{2}\sigma}{2})
\end{align*}

由于$\int^{+\infty}_{-\infty} e^{x^2} = \sqrt{\pi}$,可得到:
\begin{equation*}
    \E(X) = e^{\mu+\frac{\sigma^2}{2}}
\end{equation*}

\subsubsection{方差推导}

已知:
\begin{equation*}
    \Var(X) = \E[(X-\mu)^2] = \E(X^2 -2\mu X + \mu^2) = \E(X^2) -2\mu \E(X) + \mu^{2} = \E(X^2) - [\E(X)]^2
\end{equation*}

同上,已知对数正态分布PDF:
\begin{equation*}
    \E(X^2) = \int^{+\infty}_0 x^2 f(x) dx = \int^{+\infty}_0 \frac{x}{\sigma\sqrt{2\pi}} e^{-\frac{(\ln x - \mu)^2}{2\sigma^2}} dx
\end{equation*}

使用换元法,令$t = \frac{\ln x-\mu}{\sigma}$,则有$x = e^{\sigma t + \mu}$:
\begin{align*}
    \E(X^2) &= \frac{e^{2\mu}}{\sqrt{2\pi}} \int^{+\infty }_{-\infty } e^{-\frac{t^2}{2}+2\sigma t}dt \\
    &= \frac{e^{2\mu}}{\sqrt{2\pi}} \int^{+\infty }_{-\infty } e^{-\frac{1}{2}(t-2\sigma)^2 + 2\sigma^2}dt \qquad 
    \left(\text{对}-\frac{t^2}{2}+2 \sigma t \text{配方}\right) \\
    &= \frac{e^{2\mu+2\sigma^2}}{\sqrt{2\pi}} \sqrt{2} \int^{+\infty}_{-\infty} e^{(-\frac{t-2\sigma}{\sqrt{2}})^2} d\left(\frac{t-2\sigma}{\sqrt{2}}\right)\\
    &= e^{2\mu +2\sigma^2}
\end{align*}

此时则有:
\begin{align*}
    \Var(X) &= \E(X^2) - [\E(X)]^2  \\
    &= e^{2\mu+2\sigma^2} - (e^{\mu+\frac{\sigma^2}{2}})^2 \\
    &= e^{2\mu+2\sigma^2} - e^{2\mu+\sigma^2} \\
    &= e^{2\mu+\sigma^2}\left(e^{\sigma^2} - 1\right)\\
\end{align*}

\subsubsection{求正态分布$\mu$与$\sigma$}

已知$\ln X \sim N(\mu,\sigma^2)$、$\E(X)$与$\Var(x)$,即随机变量$X$服从对数正态分布,其对数服从正态分布,则有:
\begin{equation*}
    \mu = \ln[\E(X)] - \frac{1}{2} \ln \left[1+\frac{\Var(X)}{[\E(X)]^2}\right]
\end{equation*}
\begin{equation*}
    \sigma = \sqrt{\ln \left[1+\frac{\Var(x)}{[\E(X)]^2}\right]}
\end{equation*}


\section{BSM 偏微分方程(PDE)}

\subsection{假设}
\begin{itemize}
    \item 人性假设
          \begin{itemize}
              \item 不存在无风险套利机会(无套利)
          \end{itemize}
    \item 完美世界
          \begin{itemize}
              \item 允许卖空标的证券
              \item 没有交易费用和税收
              \item 证券交易时连续的,价格变动也是连续的
              \item 所有证券都完全可分
          \end{itemize}
    \item 可交易资产
          \begin{itemize}
              \item 证券价格遵循几何布朗运动,即$\mu$和$\sigma$为常数
              \item 衍生品有效期内,无风险利率r为常数
              \item 衍生证券有效期内,标的证券没有现金收益支付
          \end{itemize}
\end{itemize}

\subsection{推导}

假设股票价格$S_t$遵循几何布朗运动,以及其离散形式有:
\begin{align*}
    d S_t & = \mu S_t dt + \sigma S_t d Z_t \\
    \Delta S_t & = \mu S_t \Delta t + \sigma S_t \Delta Z_t
\end{align*}

假设衍生品价格$f(S_t, t)$为$S_t$以及$t$的函数,根据伊藤引理可得其连续和离散形式有:

\begin{align*}
    df(S_t,t) & = \left(\frac{\partial f}{\partial S} \mu S_t  + \frac{\partial f}{\partial t} + \frac{1}{2}\frac{\partial^2 f}{\partial S^2} \sigma^2 S_t^2 \right)dt + \frac{\partial f}{\partial S} \sigma S_t dZ_t \\
    \Delta f(S_t,t) & = \left(\frac{\partial f}{\partial S} \mu S_t  + \frac{\partial f}{\partial t} + \frac{1}{2}\frac{\partial^2 f}{\partial S^2} \sigma^2 S_t^2 \right) \Delta t + \frac{\partial f}{\partial S} \sigma S_t \Delta Z_t
\end{align*}

由此可见,股票价格与衍生品价格的风险源均来自$\Delta Z_t$,因此可以构建投资组合,由一单位衍生品空头,以及$\partial f/\partial S$单位证券多头构成,进行对冲消除该风险源:
\begin{equation*}
    \Pi_t = -f_t + \frac{\partial f}{\partial S} S_t
\end{equation*}

在$\Delta t$时间内,该投资组合价值变化$\Delta \Pi_t$为,并代入$\Delta S_t$与$\Delta f_t$:
\begin{align*}
    \Delta \Pi_t & = -\Delta f_t + \frac{\partial f}{\partial S} \Delta S_t \\
    & = -\left[ \left(\frac{\partial f}{\partial S} \mu S_t  + \frac{\partial f}{\partial t} + \frac{1}{2}\frac{\partial^2 f}{\partial S^2} \sigma^2 S_t^2 \right) \Delta t + \frac{\partial f}{\partial S} \sigma S_t \Delta Z_t \right] + \frac{\partial f}{\partial S} \left( \mu S_t \Delta t + \sigma S_t \Delta Z_t \right) \\
    & = -\left( \frac{\partial f}{\partial t} + \frac{1}{2}\frac{\partial^2 f}{\partial S^2} \sigma^2 S_t^2 \right) \Delta t \\
\end{align*}

由于此时组合消除了风险,因此组合只应获得无风险收益率:
\begin{align*}
    \Delta \Pi_t & = r \Pi_t \Delta t \\
    -\left( \frac{\partial f}{\partial t} + \frac{1}{2}\frac{\partial^2 f}{\partial S^2} \sigma^2 S_t^2 \right) \Delta t & =  r \left( -f_t + \frac{\partial f}{\partial S} S_t \right) \Delta t \\
\end{align*}

整理等式,消去$\Delta t$,即可得到\textbf{BSM偏微分方程}:
\begin{equation*}
    \frac{\partial f}{\partial t} + r S_t \frac{\partial f}{\partial S} + \frac{1}{2} \sigma^2 S_t^2 \frac{\partial^2 f}{\partial S^2} = r f_t
\end{equation*}

\section{BSM公式(鞅方法)}

在风险中性世界中,无收益资产看涨期权到期时价值的期望值为:
\begin{equation*}
    \E^\mathbb{Q}_t \left[ \max(S_T-K,0) \right]
\end{equation*}

欧式看涨期权的现值应为其期望值以无风险利率进行贴现:
\begin{equation*}
    c = e^{-r(T-t)} \E^\mathbb{Q}_t \left[ \max(S_T-K,0) \right]
\end{equation*}

同时在风险中性世界下,漂移率$\mu$应等于无风险收益率r,因此有:
\begin{equation*}
    \ln S_T - \ln S_t \sim N \left((r-\frac{\sigma^2}{2})(T-t), \sigma^2(T-t)\right)
\end{equation*}

已知:
\begin{equation*}
    S_T = S_t \exp\left[\left(r-\frac{\sigma^2}{2}\right)\left(T-t\right) + \sigma\left(Z_T - Z_t \right)\right]
\end{equation*}

已知$Y = \frac{Z_T - Z_t}{\sqrt{T-t}} \sim N(0,1)$,其密度函数为:
\begin{equation*}
    \varphi(y) = \frac{1}{\sqrt{2\pi}} e^{-\frac{1}{2}y^2}
\end{equation*}

在风险中性下的期望,可以改写为如下积分的形式:
\begin{align*}
    \E^\mathbb{Q}_t \left[ \max(S_T-K,0) \right] & = \E^\mathbb{Q}_t \left[ S_t e^{(r-\frac{1}{2}\sigma^2)(T-t) + \sigma\sqrt{T-t}Y} - K \right]^+ \\
    & = \int_{-\infty}^{\infty} \left( S_t e^{(r-\frac{1}{2}\sigma^2)(T-t) + \sigma\sqrt{T-t}y} - K \right)^+ \varphi(y) dy
\end{align*}

当$S_t e^{(r-\frac{1}{2}\sigma^2)(T-t) + \sigma\sqrt{T-t}Y} - K \geq 0$时,有$y \geq \frac{ \ln (K/S_t) - (r-\frac{1}{2}\sigma^2)(T-t)}{\sigma\sqrt{T-t}}$,设其为$-d_2$,同时假设$d_1 = d_2 + \sigma \sqrt{T-t}$。
\begin{align*}
    \E^\mathbb{Q}_t \left[ \max(S_T-K,0) \right] & = \int_{-\infty}^{\infty} \left( S_t e^{(r-\frac{1}{2}\sigma^2)(T-t) + \sigma\sqrt{T-t}y} - K \right)^+ \varphi(y) dy \\
    & = S_t e^{(r-\frac{1}{2}\sigma^2)(T-t)} \int_{-d_2}^{\infty} e^{\sigma\sqrt{T-t} y} \varphi(y)dy - K\int_{-d_2}^{\infty} \varphi(y)dy \\
    & = S_t e^{(r-\frac{1}{2} \sigma^2) (T-t)} \int_{-d_2}^{\infty}{e^{\sigma\sqrt{T-t}y} \frac{1}{\sqrt{2\pi\ }}e^{-\frac{y^2}{2}}dy} - KN\left(d_2\right) \\
    & = S_t e^{r(T-t)} \int_{-d_2}^{\infty} \frac{1}{\sqrt{2\pi}} e^{\left( -\frac{\sigma^2 (T-t)}{2} + \sigma\sqrt{T-t}y - \frac{y^2}{2} \right)} dy - KN(d_2) \\
    & = S_t e^{r(T-t)} \int_{y = -d_2}^{y = \infty} \frac{1}{\sqrt{2\pi}}e^{-\frac{\left(y-\sigma\sqrt{T-t}\right)^2}{2}}dy - KN(d_2) \quad \text{(换元法:$u =y -\sigma\sqrt{T-t}$)} \\
    & =S_t e^{r(T-t)} \int_{u = -d_2-\sigma\sqrt{T-t}}^{u = \infty} \frac{1}{\sqrt{2\pi\ }}e^{-\frac{u^2}{2}}du - KN(d_2) \quad \text{($dy = du$)} \\
    & = S_t e^{r(T-t)} \int_{-d_1}^{\infty} \frac{1}{\sqrt{2\pi}}e^{-\frac{u^2}{2}} du - KN(d_2) \\
    & = S_t e^{r(T-t)}N(d_1) - KN(d_2)
\end{align*}

得到\textbf{BSM公式},即欧式看涨期权的解析解:
\begin{align*}
    c & = e^{-r(T-t)} \E^\mathbb{Q}_t \left[ \max(S_T-K,0) \right] \\
    & = S_t N(d_1) - Ke^{-r(T-t)} N(d_2)
\end{align*}

已知期权平价公式:
\begin{equation*}
    c + K e^{r(T-t)} = p + S_t
\end{equation*}

代入BSM看涨期权解析解中,可得:
\begin{align*}
    p & = c + K e^{-r(T-t)} - S_t \\
    & = S_t N(d_1) - Ke^{-r(T-t)} N(d_2) +  Ke^{-r(T-t)} - S_t \\
    & = S_t (N(d_1) - 1) - K e^{-r(T-t)}(N(d_2)-1) \\
    & = S_t (-N(-d_1)) - K e^{-r(T-t)}(-N(-d_2)) \\
    & = K e^{-r(T-t)}N(-d_2) - S_t N(-d_1) \\
\end{align*}

此时$d_1$和$d_2$分别为:
\begin{align*}
    d_1 = \frac{\ln \frac{S_t}{K} + (r+\frac{1}{2}\sigma^2)(T-t)}{\sigma\sqrt{T-t}} \\
    d_2 = \frac{\ln \frac{S_t}{K} + (r-\frac{1}{2}\sigma^2)(T-t)}{\sigma\sqrt{T-t}}
\end{align*}

\section{内在价值(考虑中国市场的新定义)}

由于:
\begin{equation*}
    \text{期权价值(Option value)} = \text{内在价值(Intrinsic value)} + \text{时间价值(Time value)}
\end{equation*}

内在价值为即\textbf{不考虑资产价格波动}的情况下,期权条款赋予期权多头的最高价值。而时间价值为\textbf{标的资产价格波动}为期权多头(权利方)所带来的隐含价值,由于期权权利方只有权力而无义务,因此期权的时间价值应该大于0。内在价值不受时间价值的影响,因而可以使用二分法。


若定义内在价值为,期权若在当下时点到期,期权所含的的价值(Hull,CME)。这样考虑的缺点为没有考虑货币的时间价值,且在中国市场由于现货的卖空限制,其价格高于其真实价格。
\begin{align*}
    \text{看涨期权内在价值} & = \max(S_t-K) \\
    \text{看跌期权内在价值} & = \max(K-S_t)
\end{align*}

在考虑货币时间价值的情形内在价值如下,缺点为依然没有考虑中国市场的卖空限制。
\begin{align*}
    \text{看涨期权内在价值} & = \max(S_t-Ke^{-r(T-t)}) \\
    \text{看跌期权内在价值} & = \max(Ke^{-r(T-t)}-S_t)
\end{align*}

因此考虑使用期货价格代替现货价格,以为期货市场多空双方均能自由表达其看法,因此有:
\begin{align*}
    \text{看涨期权内在价值} & = \max((F_{t,T}-K)e^{-r(T-t)}) \\
    \text{看跌期权内在价值} & = \max((K-F_{t,T})e^{-r(T-t)})
\end{align*}

由于在中国市场ETF期权有红利保护机制,即会下调行权价格,放大每手期权数量,相当于变相抬高了股票价格,或复权(加挂A标记的期权)。且在ETF中的成分股分红,其分红留在ETF当中。而ETF没有期货,只有股指期货,而股指期货不对分红进行调整,即没有红利保护,即其成分股分红后股指自然下跌。因而在使用股指期货或期权以及ETF现货或期权时,需要做红利调整。即在ETF现货中将红利剔除,此时有:
\begin{equation*}
    F_{t,T} = (S_t-I)e^{r(T-t)}
\end{equation*}

此时则有,将上式代入,在中国市场中:
\begin{align*}
    \text{看涨期权内在价值} & = \max((F_{t,T}-K)e^{-r(T-t)}+I) \\
    \text{看跌期权内在价值} & = \max((K-F_{t,T})e^{-r(T-t)}-I)
\end{align*}

因为平值点为使内在价值为零,则平值点定义为为$F_{t,T}=K$,这样定义使得实值虚值部分左右较为对称,有利于比较。此时有当$F<K$为OTM,此时值域为正,当$F>K$为ITM,则有值域为负。此时有对数在值状态(log-moneyness):
\begin{equation*}
    \ln{\frac{K}{K_{atm}}} = \ln{\frac{K}{F}}
\end{equation*}

同时可以发现,在PCP下:
\begin{equation*}
    c = p + (F_{t,T}-K) e^{-r(T-t)}
\end{equation*}

对于平直期权ATM,则有$F_{t,T}=K$,易得此时$c=p$。而当看涨期权为ITM,其内在价值部分不为零。而对于此时得看跌期权为OTM,其内在价值为零,而仅有时间价值,因此可以得到,在新平值点定义下的,相同行权价,相同期限的看涨看跌期权:
\begin{equation*}
    c_{\text{时间价值}} = p_{\text{时间价值}}
\end{equation*}


\section{平价期权}

当平值点为$S = K e^{-r(T-t)}$时,将其带入看涨BSM公式当中,则有:
\begin{equation}
    \frac{c}{S} = N(d_1) - N(d_2)
\end{equation}

对于看跌期权则有:
\begin{align*}
    \frac{p}{S} & = N(-d_2) - N(-d_1) \\
    & = 1 - N(d_2) - [1-N(d_1)] \\
    & = N(d_1) - N(d_2) = \frac{c}{S}
\end{align*}

对于$d_1$和$d_2$,此时有:
\begin{equation}
    d_1 = \frac{ln(S/K)+(r+\sigma^2/2)(T-t)}{\sigma\sqrt{T-t}} = \frac{\sigma}{2}\sqrt{T-t}
\end{equation}

\begin{equation}
    d_2 = \frac{ln(S/K)+(r-\sigma^2/2)(T-t)}{\sigma\sqrt{T-t}} = d_1 - \sigma\sqrt{T-t}= -\frac{\sigma}{2}\sqrt{T-t}
\end{equation}

则对于欧式平价期权:
\begin{align*}
    \frac{c}{S} & = \frac{p}{S} = N \left( \frac{\sigma\sqrt{T-t}}{2} \right) - N\left( -\frac{\sigma\sqrt{T-t}}{2} \right) \\
    & = 2N\left( \frac{\sigma\sqrt{T-t}}{2} \right) - 1 \\
    & = 2\left[\frac{1}{2} + \frac{1}{\sqrt{2\pi}} \left(\frac{\sigma\sqrt{T-t}}{2} - \frac{(\sigma\sqrt{T-t}/2)^3}{6} + \frac{(\sigma\sqrt{T-t}/2)^5}{40} - \dots + \dots\right) \right] - 1 \qquad (\text{使用泰勒展开}) \\
    & \approx \frac{\sigma\sqrt{T-t}}{2\pi} \approx 0.4\sqrt{T-t}
\end{align*}

\section{波动率}

为人们对未来给定期限的波动率的预期值
\begin{itemize}
    \item 历史波动率(Historical volatility):使用过去代替未来
          \begin{itemize}
              \item 样本对数收益率标准差(日频数据)
              \item 已实现波动率(Realized volatility,日内高频,5分钟,假设均值为零)
              \item 极差波动率
          \end{itemize}
    \item 历史波动率(Historical volatility):利用历史数据进行建模,并且预测
          \begin{itemize}
              \item 广义自回归条件异方差(GARCH,计量方法)
              \item 随机波动率(Sochastic volatility,随机过程)
          \end{itemize}
    \item 隐含波动率(Implied volatility):直接从期权价格中提取未来预期
\end{itemize}

\section{注意与备注}

\begin{itemize}
    \item 期限、无风险利率、波动率应匹配(以年为单位,一般使用交易日计算,美国交易日252天)
    \item 无风险利率选择即期利率(Spot rate)而非到期收益率(YTM,真实收益率,票息5\%,但非平价发行)
    \item 由于只有交易日才有历史数据与收益率数据,波动率使用交易天数进行年化,中国240天左右,美国252天
    \item 波动率为一个时间窗口(一般为年,252天交易日,较以月每21天为窗口更为平滑)内连续复利收益率或对数收益率($\ln S_t/S_{t-1}$)标准差进行年化。即日频波动率乘以$\sqrt{252}$(一天的方差为$s^2$,由于方差可加,252个交易日的方差即为$s^2 \times 252$,标准差或波动率为$s\sqrt{252}$),月频波动率应乘以$\sqrt{252/21}$
\end{itemize}

\subsection{比例收益率与对数收益率}

股票价格服从几何布朗运动:
\begin{equation*}
    dS_t = \mu S_t dt + \sigma S_t dZ_t
\end{equation*}

其离散形式可写作:
\begin{equation*}
    \frac{\Delta S_t}{S_t} = \mu \Delta t + \sigma \varepsilon \sqrt{\Delta t}
\end{equation*}

其期望有,可以看到$\Delta S_t/S_t$为$\Delta t$时间内百分比年化收益率或\textbf{比例收益率}(percentage returns)为$\mu$:
\begin{equation*}
    \E(\frac{\Delta S_t}{S_t}) = \mu \Delta t
\end{equation*}

而连续复利收益率或\textbf{对数收益率}(log returns)的期望则为:
\begin{align*}
    d\ln S       & = (\mu-\frac{\sigma^2}{2})dt+\sigma dZ_t \\
    \E(d\ln S_t) & = (\mu - \frac{\sigma^2}{2})dt
\end{align*}

比例收益率在实际应用过程中意义较小,假设4年盈亏为$+50\%$,$-50\%$,$+50\%$,$-50\%$,其比例收益率期望与均值$\mu$均为0,但实际上相比期起初有-43.75\%的亏损。而使用几何平均(复利)计算,年化亏损$-13.40\%$即盈亏应使用几何平均的方式计算,简单的算术平均比例收益率没有意义。而使用对数收益率,其期望为$\mu-\sigma^2/2$,即算术平均$\mu$需要减去$\sigma^2/2$,才是几何平均期望。在此例子中均值为0,方差为0.25,此时对数收益率的期望为$-12.5\%$。即波动越大,降低实际收益率,符合现实情况,具有经济学意义。

\subsection{做空限制}
且在中国市场中现货存在较大的做空限制,即在现货市场的价格由看多者和少量看空者决定,并不能反应所有投资者的真实情绪,以至于难以复制期权,违法BSM公式假设条件。解决方法有:


\begin{enumerate}
    \item 使用期货进行贴现,得到其隐含现货价格,使用BSM进行计算,其中有:
          \begin{itemize}
              \item 期货隐含现货价格
                    \begin{equation*}
                        S^* = F e^{-q(T-t)}
                    \end{equation*}
              \item 期权隐含现货价格
                    \begin{equation*}
                        S^* = (c-p) + Ke^{-r(T-t)}
                    \end{equation*}
          \end{itemize}
    \item 直接使用Black公式,使用期货价格进行计算,即:
          \begin{equation*}
              c = e^{-r(T-t)} \left[ F_t N(d_1)-K N(d_2) \right]
          \end{equation*}
\end{enumerate}



\end{document}