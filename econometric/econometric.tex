\documentclass[11pt]{article}
\usepackage[a4paper,left=6em,right=6em]{geometry}
\usepackage{amsmath,amsfonts,amsthm,bbold}
\usepackage{booktabs,float,multirow}
\usepackage{cancel}
\usepackage{enumitem}
\usepackage{multicol}
\usepackage{graphicx}
\usepackage[toc,title]{appendix}
\usepackage{fancyhdr}
\pagestyle{fancy}
\setlength{\headheight}{15pt}
\usepackage{footmisc}
\usepackage{hyperref}
\usepackage{tocloft}
\hypersetup{
    colorlinks=true, %set true if you want colored links
    linkcolor=blue,
    linktoc=all, %set to all if you want both sections and subsections linked
    citecolor=black,
    filecolor=black,
    urlcolor=black
}
\usepackage[UTF8]{ctex}

\setlength{\cftbeforesecskip}{6pt}
\setlength{\parskip}{0.6em}
\renewcommand{\baselinestretch}{1.3}
\setlist{noitemsep,itemindent=1em,topsep=0em,leftmargin=4em,rightmargin=4em}
\setlist[2]{leftmargin=2em}
\newcommand{\divider}{\vspace{-\parskip}\noindent\rule{\linewidth}{0.4pt}}

\newtheorem{thm}{定理}[section] 
\newtheorem{defi}[thm]{定义}
\newtheorem{lemm}[thm]{引理}
\newtheorem{coro}[thm]{推论}

\newcommand{\E}{\mathbb{E}}
\newcommand{\rnE}{\widetilde{\mathbb{E}}}
\newcommand{\wt}[1]{\widetilde{#1}}
\DeclareMathOperator{\Var}{Var}
\DeclareMathOperator{\Cov}{Cov}
\newcommand{\abs}[1]{\left\lvert #1\right\rvert}
\newcommand{\norm}[1]{\left\lVert #1\right\rVert}

\title{计量经济学}
\author{杨弘毅}
\date{创建: 2020 年 4 月 9 日 \\修改: \today}

\begin{document}
\maketitle

\tableofcontents

\section{基础}

\subsection{期望}

对于随机变量$X$,其概率空间为$(\Omega,\mathcal{F},P)$,期望值$\E[X]$,应有:
\begin{equation*}
    \E[X] = \int_{\Omega} X(\omega) dP(\omega)
\end{equation*}

在离散以及连续情形下有如下定义,其中$f(x)$为变量$X$的概率密度函数(PDF)。
\begin{align*}
    \E[X] &= \sum_{i=i}^n x_i p_i =x_1 p_1 + x_2 p_2 + \dots + x_n p_n \\
    \E[X] &= \int x f(x) dx
\end{align*}

其性质有:
\begin{align*}
    \E[X+Y] &= \E[X] + \E[Y] \\
    \E[aX] &= a \E[X] \\
    \E[XY] &= \E[X]\E[Y] \quad \text{(X,Y are independent)}
\end{align*}


\subsection{方差}

对于方差(Variance),定义有:
\begin{align*}
    \Var(X) &= \Cov(X,X) = \sigma_X^2\\ 
    &= \E[(X-\E[X])^2] \\
    &= \E[X^2 - 2X\E[X] + \E[X]^2] \\
    &= \E[X^2] - 2\E[X]^2 + \E[X]^2 \\
    &= \E[X^2] - \E[X]^2
\end{align*}

其性质有:
\begin{align*}
    \Var(X + a) &= \Var(X) \\
    \Var(aX) &= a^2\Var(X) \\
    \Var(aX \pm bY) &= a^2\Var(X) + b^2\Var(Y) \pm 2ab \Cov(X,Y) \\
    \Var(\sum_{i=1}^{N} X_i) &= \sum_{i,j=1}^{N}\Cov(X_i,X_j) = \sum_{i=1}^{N}\Var(X_i) + \sum_{i \neq j}\Cov(X_i,X_j) \\
    \Var(\sum_{i=1}^{N} a_i X_i) &= \sum_{i,j=1}^{N}a_i a_j\Cov(X_i,X_j) \\
    &= \sum_{i=1}^{N}a_i^2\Var(X_i) + \sum_{i \neq j} a_i a_j \Cov(X_i,X_j) \\
    &= \sum_{i=1}^{N}a_i^2\Var(X_i) + 2\sum_{1 \leq i \leq j \leq N} a_i a_j \Cov(X_i,X_j)
\end{align*}

\subsection{协方差}

对于协方差(Covariance)其定义有:
\begin{align*}
    \Cov(X,Y) &= \E[(X-E(X))(Y-E(Y))] \\
    &= \E[XY - X\E[Y] -Y\E[X]+ \E[X]\E[Y]] \\
    &= \E[XY] - \E[X]\E[Y] - \E[X]\E[Y] + \E[X]\E[Y] \\
    &= \E[XY] - \E[X]\E[Y]
\end{align*}

性质有:
\begin{align*}
    \Cov(X,a) &= 0 \\
    \Cov(X,X) &= \Var(X) \\
    \Cov(X,Y) &= \Cov(Y,X) \\
    \Cov(aX,bY) &= ab\Cov(X,Y) \\
    \Cov(X+a,Y+b) &= \Cov(X,Y) \\
    \Cov(aX+bY,cW+dV) &= ac\Cov(X,W)+ad\Cov(X,V)+bc\Cov(Y,W)+bd\Cov(Y,V)
\end{align*}

\subsection{相关系数}

相关系数(Correlation Coefficient),为研究变量间线性相关程度的量。最早由统计学家卡尔·皮尔逊设计,也称为皮尔逊积矩相关系数(Pearson product-moment correlation coefficient),或皮尔逊相关系数:
\begin{equation*}
    \rho_{X,Y} = \frac{\Cov(X,Y)}{\sigma_X\sigma_Y}
    = \frac{\E[(X-\E[X])(Y-\E[Y])]}{\sigma_X\sigma_Y}
\end{equation*}

\section{假设检验(Statistical hypothesis testing)}

\textbf{原假设($\text{H}_0$,null hypothesis)},也称为零假设或虚无假设。而与原假设相反的假设称为\textbf{备择假设($\text{H}_a$,althernative hypothesis)}。假设检验的核心为\textbf{反证法}。在数学中,由于不能穷举所有可能性,因此无法通过举例的方式证明一个命题的正确性。但是可以通过举一个反例,来证明命题的错误。在掷骰子的例子中,在每次掷的过程相当于一次举例,假设进行了上万次的实验,即便实验结果均值为3.5,也无法证明总体的均值为3.5,因为无法穷举。

可以理解为原假设为希望拒绝的假设,或反证法中希望推翻的命题。我们先构造一个小概率事件作为原假设($\text{H}_0$),并假设其正确。如样本均值等于某值,两个样本均值是否相等,样本中的不同组直接是否等概率发生,一般使用等式(小概率)作为原假设。如果抽样检验中小概率事件发生,则说明原假设的正确性值得怀疑。如此时假设实验的结果(样本)远大于或小于理论计算结果3.5,即发生了小概率事件,那么就有理由相信举出了一个反例,这时就可以否定原命题(reject the null hypothesis)。而相反,如果原假设认为均值为3.5,在实验的过程中结果大概率不会偏离这个理论值太多,可以认为我们并没办法举出反例。由于不能直接证明原命题为真,只能说”We can not(fail to) reject the null hypothesis“,无法拒绝原命题。

在需要评估总体数据的时候,由于经常无法统计全部数据,需要从总体中抽出一部分样本进行评估。假设掷骰子一个骰子,其期望为3.5,但假设掷骰子了100次,计算均值为3.47,由于总体的理论值和样本呢的实验值可能存在偏差,误差永远存在,无法避免。那么是否可以认为么3.47“等于”3.5?这时候就需要要界定一个\textbf{显著水平($\alpha$,significant level)},相当于设定一个等于的阈值范围。即多小概率的事情发生,是$10\%$还是$5\%$的概率,使我们认为举出了一个反例,值得去怀疑原命题的正确性。当我们知道随机变量的分布时候,根据所进行的检验,我们可以根据计算出的\textbf{统计量(test statistic)},由于分布已知,统计量对应了一个\textbf{p值(p-value)},即小概率(极端)事件发生的概率,因此在图形上表示为统计量向两侧延申的线下区域。如果这个概率足够低,如小于$\alpha=5\%$,那么就有理由拒绝原假设。

用1-显著水平($1-\alpha$),得到值称为\textbf{置信水平(confidence level)}(概率大小)。置信水平越大,对应的置信区间也越大(随机变量范围)。此时有置信水平为$1-\alpha$,假设置信区间为$(a,b)$,那么有$P(a<\text{随机变量}<b)=1-\alpha$。对于双侧检验,有置信水平为$1-\alpha$(概率大小),两侧拒绝域分别为$\alpha/2$。对于单侧检验,则有单侧拒绝域大小为$\alpha$。

\section{Chi-square distribution}

假设有随机变量$X$服从标准正态分布,即有$X \sim N(0,1)$,此时有随机变量$Q_1=X^2$,则有随机变量$Q_1$服从卡方分布($\chi^2\text{-distribution}$),由于此时只有一个随机变量,因此卡方分布自由度(degree of freedom)为1,即$Q_1 \sim \chi^2(1)$。如随机变量$Q_2 = X_1^2 + X_2^2$,且$X_1$与$X_2$同时服从标准正态分布。则此时$Q_2$服从自由度为2的卡方分布,即$Q_2 \sim \chi^2(2)$。


\subsection*{Goodness of fit}

Pearson's chi-squared test
\begin{equation*}
    \chi^2 = \sum_i^n \frac{(O_i - E_i)^2}{E_i}
\end{equation*}

\begin{list}{-}{}
    \item $O_i$ the number of observations of type i
    \item $E_i$ the expected(theoretical) number of type i
\end{list}

\section{Probability vs Likelihood}

\subsection{Probability}

P( data | distribution ) = area under curve

P( weight between 32g and 34g | mean = 32 and standard deviation = 2.5) = 0.29

P( weight > 34g | mean = 32 and standard deviation = 2.5) = 0.21

\subsection{Likelihood}

L( distribution | data ) = value of the curve (y)

L( mean = 32 and standard deviation = 2.5 | mouse weights 34g ) = 0.12

L( mean = 34 and standard deviation = 2.5 | mouse weights 34g ) = 0.21

在调整了分布的mean之后,likelihood最大,在mean=34 sigma=2.5的正态分布中,抽中一只34g的老鼠的概率最大

\subsection{Maximum likelihood}

测量了数只老鼠的重量,尝试找到其分布,miximizes the likelihood 找到最大化所有观察重量likelihood的分布,找到mean 和standard deviation




\end{document}