\documentclass[11pt]{article}
\usepackage{amsmath,amsthm}
% \usepackage{amstext}
\usepackage[a4paper,left=10mm,right=10mm,top=15mm,bottom=15mm]{geometry}
\usepackage{booktabs}
\usepackage{float}
\usepackage[UTF8]{ctex}

\newtheorem{theorem}{Theorem}[section]
\newtheorem{lemma}[theorem]{Lemma}

\title{希腊字母}
\author{杨弘毅}
\date{创建: 2020 年 3 月 20 日 \\修改: \today}

\begin{document}
\maketitle

\section{Pre}
\subsection{Normal Distribution}
$N(x)$为标准正态分布(Standard Normal)的累积分布函数(CDF,Cumulative Distribution Function):
\begin{equation*}
    N(x) = \frac{1}{\sqrt{2\pi}} \int_{-\infty}^{x}e^{-z^2/2}dz
\end{equation*}

同时由于正态分布的对称性,易知
\begin{equation*}
    N(-x) = 1-N(x)
\end{equation*}

$N'(x)$为标准正态分布的概率密度函数(PDF,Probability Density Function):
\begin{equation*}
    N'(x) = \frac{1}{\sqrt{2\pi}} e^{-x^2/2}
\end{equation*}

同样由于正态分布的对称性,易知:
\begin{equation*}
    N'(x) = N'(-x)
\end{equation*}

\subsection{Greeks}
\begin{table}[H]
\centering
\begin{tabular}{@{}rl@{}}
\toprule
\multicolumn{2}{c}{Greeks} \\ 
\midrule
Delta  & $\delta = \tfrac{\partial V}{\partial S}$ \\
Gamma  & $\gamma = \tfrac{\partial^2 V}{\partial S^2}$ \\
Theta  & $\theta = \tfrac{\partial V}{\partial t}$ \\
Rho    & $\rho = \tfrac{\partial V}{\partial r}$ \\
Vega   & $\nu = \tfrac{\partial V}{\partial\sigma}$ \\
\bottomrule
\end{tabular}
\end{table}

\section{Black formula}

\begin{align*}
    c &= e^{-r\tau}[FN(d_1) - KN(d_2)] \\
    p &= e^{-r\tau}[KN(-d_2) - FN(-d_1)]
\end{align*}

其中有:
\begin{align*}
    d_1 &= \frac{\ln(F/K) + \sigma^2/2\tau}{\sigma \sqrt{\tau}} \\
    d_2 &= \frac{\ln(F/K) + \sigma^2/2\tau}{\sigma \sqrt{\tau}} \\
    d_2 &= d_1 - \sigma \sqrt{\tau}
\end{align*}

\begin{equation*}
    \frac{\partial d_1}{\partial F} = \frac{\partial d_2}{\partial F} = \frac{\frac{\partial \ln(F/K)}{\partial F} \sigma \sqrt{\tau}}{(\sigma \sqrt{\tau})^2} = \frac{\partial (\ln F - \ln K)/\partial F}{\sigma\sqrt{\tau}} = \frac{1}{F\sigma\sqrt{\tau}}
\end{equation*}

\subsection{Delta}
\begin{align*}
    FN'(d_1) &= \frac{F}{\sqrt{2\pi}}e^{-d_1^2/2} \\
    KN'(d_2) &= \frac{K}{\sqrt{2\pi}}e^{-d_2^2/2} = \frac{K}{\sqrt{2\pi}}e^{-d_1^2/2+d_1\sigma\sqrt{\tau}-\sigma^2\tau/2} \\
    &= \frac{K}{\sqrt{2\pi}}e^{-d_1^2/2+\ln(F/K)} 
    \qquad \left(d_1\sigma\sqrt{\tau} = \ln(F/K)+\sigma^2\tau\right) \\
    &= \frac{F}{\sqrt{2\pi}}e^{-d_1^2/2} \\
    &= FN'(d_1) \\
    \text{Delta}_c &= e^{-r\tau}[N(d_1) + FN'(d_1)\frac{\partial d_1}{\partial F} - KN'(d_2)\frac{\partial d_2}{\partial F}] \\
    &= e^{-r\tau}N(d_1) \\
    \text{Delta}_p &= e^{-r\tau}[KN'(-d_2)\frac{\partial d_2}{\partial F} - N(-d_1) - FN'(-d_1)\frac{\partial d_1}{\partial F}] \\
    &= e^{-r\tau}(-N(-d_1)) \qquad \left(KN'(-d_2) = KN'(d_2),\, FN'(-d_1)= FN'(d_1)\right) \\
    &= e^{-r\tau}(N(d_1)-1)
\end{align*}

\section{BSM formula}
\begin{align*}
    C_t &= S_te^{-q\tau}N(d_1) - Ke^{-r\tau}N(d_2) \\
    P_t &= -S_te^{-q\tau}N(-d_1) + Ke^{-r\tau}N(-d_2)
\end{align*}

其中有:
\begin{align*}
    d_1 &= \frac{\ln(S/K) + (r+\frac{\sigma^2}{2})\tau}{\sigma \sqrt{\tau}} \\
    d_2 &= \frac{\ln(S/K) + (r-\frac{\sigma^2}{2})\tau}{\sigma \sqrt{\tau}} \\
    d_2 &= d_1 - \sigma \sqrt{\tau}
\end{align*}

\begin{lemma}
    \begin{equation*}
        \frac{\partial d_2}{\partial S} = \frac{\partial d_1}{\partial S} = \frac{\frac{\partial \ln(S/K)}{\partial S} \sigma \sqrt{\tau}}{(\sigma \sqrt{\tau})^2} = \frac{\partial (\ln S - \ln K)/\partial S}{\sigma\sqrt{\tau}} = \frac{1}{S\sigma\sqrt{\tau}}
    \end{equation*}
    \begin{equation*}
        \frac{\partial d_2}{\partial \tau} = \frac{\partial d_1}{\partial \tau} - \frac{1}{2}\frac{\sigma}{\tau}
    \end{equation*}
    \begin{equation*}
        \frac{\partial d_2}{\partial \sigma} = \frac{\partial d_1}{\partial \tau} - \sqrt{\tau}
    \end{equation*}
    \begin{equation*}
        \frac{\partial d_2}{\partial r} = \frac{\partial d_1}{\partial r}
    \end{equation*}
\end{lemma}

\begin{lemma}
    对于BSM公式,有:
    \begin{equation*}
        SN'(d_1) = Ke^{-r\tau} N'(d_2)
    \end{equation*}
\end{lemma}

\begin{proof}
    已知:
    \begin{align*}
        d_2^2-d_1^2 &= (d_2-d_1)(d_2+d_1) \\
        &= (-\sigma\sqrt{\tau})(2d_1-\sigma\sqrt{\tau}) \\
        &= (-\sigma\sqrt{\tau})\left( \frac{2\ln(S/K) + 2(r+\sigma^2/2)\tau}{\sigma\sqrt{\tau}} -\sigma\sqrt{\tau}\right) \\
        &= -2\left[\ln\frac{S}{K}+r\tau\right]
    \end{align*}

    则有:
    \begin{equation*}
        \ln\left(\frac{N'(d_1)}{N'(d_2)}\right)
        = -\frac{d_1^2}{2} + \frac{d_2^2}{2}
        = \frac{1}{2} (d_2^2-d_1^2)
        = -\left[\ln\frac{S}{K} +r\tau\right]
    \end{equation*}

    对等式两边取指数:
    \begin{align*}
        \frac{N'(d_1)}{N'(d_2)} 
        &= \exp\left( -\left[\ln\frac{S}{K} +r\tau\right] \right) \\
        &= \exp \left( \ln\frac{K}{S}-r\tau \right) \\
        &= \frac{K}{S} e^{-r\tau} \\
        SN'(d_1) &= Ke^{-r\tau} N'(d_2)
    \end{align*}
\end{proof}

\subsection{Delta}
\noindent
\begin{align*}
    \text{Delta}_c &= \frac{\partial C}{\partial S}
    = N(d_1) + \frac{1}{S\sigma\sqrt{\tau}} \left[ SN'(d_1)-Ke^{-r\tau}N'(d_2) \right] = N(d_1) \\
    \text{Delta}_p &= \frac{\partial P}{\partial S}
    = -N(-d_1) + \frac{1}{S\sigma\sqrt{\tau}} \left[ SN'(d_1)-Ke^{-r\tau}N'(d_2) \right] = N(d_1)-1
\end{align*}

对于已知$\partial C/\partial S$,可对PCP求导:

\subsection{Gamma}
对于gamma而言,calls和puts相同
\begin{equation*}
    \text{Gamma} = \frac{\partial^2 V}{\partial S^2} = \frac{N'(d_1)}{S\sigma\sqrt{\tau}}
\end{equation*}


\subsection{Vega}
对于vega而言,calls和puts相同

\begin{equation*}
    \text{Vega} = \frac{\partial V}{\partial \sigma}  = SN'(d_1)\sqrt{\tau}
\end{equation*}

\begin{align*}
    \text{Vega} &= \frac{\partial C}{\partial \sigma} = \frac{\partial}{\partial \sigma} \left[S_t N(d_1) - K e^{-r\tau}N(d_2)\right] \\
    &= S_t N'(d_1) \frac{\partial d_1}{\partial \sigma} - Ke^{-r\tau}N'(d_2) \frac{\partial d_2}{\partial \sigma} \\
    &= S_t N'(d_1) \frac{\partial d_1}{\partial \sigma} - Ke^{-r\tau}N'(d_2) \left[ \frac{\partial d_1}{\partial \sigma} -\sqrt{\tau} \right] \qquad \text{(对$d_2 = d_1 - \sigma\sqrt{\tau}$求导)} \\
    &= \left[ S_t N'(d_1) - Ke^{-r\tau}N'(d_2) \right] \frac{\partial d_1}{\partial \sigma} + Ke^{-r\tau}N'(d_2) \sqrt{\tau} \\
    &= S_t N'(d_1)\sqrt{\tau} \qquad \text{(已知$S_t N'(d_1) = Ke^{-r\tau}N'(d_2)$)}
\end{align*}

\end{document}