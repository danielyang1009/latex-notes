\ifx \authorname\undefined
  \def\authorname{杨弘毅}
\else
\fi

\author{\authorname}
\date{创建:\startdate \\修改:\today}

\usepackage[a4paper,left=6em,right=6em]{geometry}
\usepackage{amsmath,amsfonts,amsthm,bbold,bm}
\usepackage{booktabs,float,multirow}
\usepackage{cancel}
\usepackage{enumitem}
\usepackage{multicol}
\usepackage{graphicx}
\usepackage[toc,title]{appendix}
\usepackage{tikz}
\usetikzlibrary{arrows.meta}
\usetikzlibrary{patterns}
\usetikzlibrary{decorations.pathreplacing}
\usetikzlibrary{decorations.pathmorphing}
\usepackage{subcaption}
\usepackage{fancyhdr}
\pagestyle{fancy}
\setlength{\headheight}{15pt}
\usepackage{footmisc}
\usepackage{hyperref}
\usepackage{tocloft}
\hypersetup{
    colorlinks=true, %set true if you want colored links
    linkcolor=blue,
    linktoc=all, %set to all if you want both sections and subsections linked
    citecolor=black,
    filecolor=black,
    urlcolor=blue
}
\usepackage[UTF8]{ctex}
\usepackage{CJKfntef}
% \usepackage{xeCJKfntef}

\title{\articlename}

% Format
\setlength{\cftbeforesecskip}{6pt}
\setlength{\parskip}{0.6em}
\renewcommand{\baselinestretch}{1.4}
\setlist{noitemsep,itemindent=1em,topsep=0em,leftmargin=4em,rightmargin=4em}
\setlist[2]{leftmargin=2em}

% Shortcut
\newcommand{\divider}{\vspace{-\parskip}\noindent\rule{\linewidth}{0.4pt}}
\newcommand{\tops}[1]{\texorpdfstring{#1}{TEXT}}

% Theorem
\newtheorem{thm}{定理}[section] 
\newtheorem{proposition}[thm]{命题}
\newtheorem{lemma}[thm]{引理}
\newtheorem{corollary}[thm]{推论}
\newtheorem{property}[thm]{性质}
\newtheorem{example}[thm]{例子}
\newtheorem{remark}[thm]{备注}
\newtheorem{note}[thm]{注释}

% Symbol
\newcommand{\E}{\mathbb{E}}
\newcommand{\mcl}{\mathcal{L}}
\newcommand{\rnE}{\widetilde{\mathbb{E}}}
\newcommand{\wt}[1]{\widetilde{#1}}
\DeclareMathOperator{\Var}{Var}
\DeclareMathOperator{\Cov}{Cov}
\newcommand{\abs}[1]{\left\lvert #1\right\rvert}
\newcommand{\norm}[1]{\left\lVert #1\right\rVert}
\newcommand{\given}{\,\vert\,}
% \newcommand{\hyphen}{\hbox{-}}

% Number sets
\newcommand{\C}{\mathbb{C}}
\newcommand{\R}{\mathbb{R}}
\newcommand{\Q}{\mathbb{Q}}
\newcommand{\Z}{\mathbb{Z}}
\newcommand{\N}{\mathbb{N}}

% Matrix
\newcommand{\jaco}{\mathbf{J}}
\newcommand{\hess}{\mathbf{H}}