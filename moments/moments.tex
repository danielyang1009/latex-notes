\documentclass[11pt]{article}
\usepackage[a4paper,left=20mm,right=20mm,top=15mm,bottom=15mm]{geometry}
\usepackage{amsmath,amstext,amsthm}
\usepackage{booktabs,multirow,float}
\usepackage{hyperref}
\hypersetup{
    colorlinks=true, %set true if you want colored links
    linkcolor=blue,
    linktoc=all, %set to all if you want both sections and subsections linked
    citecolor=black,
    filecolor=black,
    urlcolor=black
}
\usepackage[UTF8]{ctex}
% ctex与amsmath冲突,放最后可解决

\newcommand{\E}{\operatorname{E}}
\newcommand{\Var}{\operatorname{Var}}
\newcommand{\Cov}{\operatorname{Cov}}

\newtheorem{thm}{定理}[subsection]

\title{Moment}
\author{杨弘毅}
\date{创建: 2021 年 4 月 8 日 \\修改: \today}
\begin{document}
\maketitle

\tableofcontents

\section{定义}

\subsection{含义}
数学中矩的概念来自物理学,在物理学中,矩表示距离和物理量的乘积。如力与力臂(参考点的距离)的乘积,得到的是力矩(或扭矩)。可以理解为一杆“秤”,“秤”的平衡的两边重量与距离的乘积相同,则能保持平衡。

而在概率论上,可以理解秤为一杆秤的两端的概率为1,中心点概率为0。如一端秤砣重量,为中奖金额$500$元,但中奖概率为千分之一,即离中心点距离为$0.1\%$,那么期望为$0.5$元。可以理解为了使得秤保持平衡,则另一端,在概率为1,其秤砣重量,中奖金额应为$0.5$元。

\subsection{期望}

这样既可以把期望看成是矩,即距离(概率)乘以力(随机变量)的大小。对于$n$阶矩即对$x^n$q求期望,在离散形式下有:
\begin{equation*}
    E[x] = \sum_i p_i x_i
\end{equation*}

在连续形式下,n阶矩可以表示为$(x-c)^n$的期望,其中$f(x)$为概率密度函数(probability density function):
\begin{equation*}
    \mu_n = \int_{-\infty}^{\infty} (x-c)^n f(x) dx
\end{equation*}

\begin{table}[ht!]
\centering
\begin{tabular}{@{}cll@{}}
\toprule
阶(Order) & \multicolumn{1}{c}{非中心矩(Non-central)} & \multicolumn{1}{c}{中心矩(Central)} \\ \midrule
1st & $\E(x)=\mu $ & \\
2nd & $\E(x^2) $ & $\E[(x-\mu)^2]$   \\
3rd & $\E(x^3) $ & $\E[(x-\mu)^3]$   \\
4th & $\E(x^4) $ & $\E[(x-\mu)^4]$   \\ \bottomrule
\end{tabular}
\end{table}

常用的有一至四阶矩:
\begin{itemize}
    \setlength{\itemsep}{0em}
    \item 均值 $\text{Mean}(x)$ 为一阶中心矩
    \item 方差 $\text{Variance}(x) = \E(x-\mu)^2$ 为二阶非中心矩
    \item 偏度 $\text{Skewness}(x) = \frac{\E[(x-\mu)^4]}{\sigma^3}$ 为三阶标准矩 
    \item 峰度 $\text{Kurtosis}(x) = \frac{\E[(x-\mu)^4]}{\sigma^4}$ 为四阶标准矩
\end{itemize}

\subsection{原点矩(Raw/crude moment)}

当$c=0$时,称为原点矩。此时则有\textbf{平均数(mean)}或\textbf{期望(expected value)}的连续形式为:
\begin{equation*}
    \mu = E(x) = \int_{-\infty}^{\infty} (x-0)^1 f(x) dx =
    \int_{-\infty}^{\infty} x f(x) dx
\end{equation*}

其离散形式为:
\begin{equation*}
    \mu = E(x) = \sum_i x_i p_i
\end{equation*}

\subsection{中心矩(Central moment)}

期望值可以成为随机变量的中心,即当$c=E(x)$时
\begin{equation*}
    \mu_n = E[(x-E(x))^n] = \int_{-\infty}^{\infty} (x-E(x))^n f(x) dx
\end{equation*}

同时可知任何变量的一阶中心矩为0:
\begin{align*}
    \mu_1 &= \int_{-\infty}^{\infty} (x-E(x))^1 f(x) dx \\
    &= \int_{-\infty}^{\infty} x f(x) dx - \int_{-\infty}^{\infty} E(x) f(x) dx \\
    &= E(x) - E(x) \int_{-\infty}^{\infty} f(x) dx \\
    &= E(x) - E(x) \times 1 = 0 
\end{align*}

而二阶中心矩(second central moment)为\textbf{方差(Variance)}
\begin{align*}
    \mu_2 &= \int_{-\infty}^{\infty} (x-E(x))^2 f(x) dx \\
    &= \int_{-\infty}^{\infty} x^2 f(x)dx - 2 E(x) \int_{-\infty}^{\infty} x f(x)dx + [E(x)]^2\int_{-\infty}^{\infty}f(x)dx \\
    &= \int_{-\infty}^{\infty} x^2 f(x)dx - 2 E(x) E(x) + [E(x)]^2\times 1 \\
    &= \int_{-\infty}^{\infty} x^2 f(x)dx - [E(x)]^2 \\
    &= E(x^2) - [E(x)]^2 = \sigma^2
\end{align*}

其离散形式则有:
\begin{equation*}
    \text{Var}(x) = \sigma^2 = \sum p_i (x_i - \mu)^2 
\end{equation*}

\subsection{标准矩(Standardized moment)}

标准矩为标准化(除以标准差)后的中心矩,第$n$阶中心矩(standardized moment of degree n)有:
\begin{equation*}
    \mu_n = E[(x-\mu)^n] = \int_{-\infty}^{\infty} (x-\mu)^n f(x)dx
\end{equation*}

已知标准差的$n$次方有:
\begin{equation*}
    \sigma^n = \left(\sqrt{E[(x-\mu)^2]}\right)^n = (E[(x-\mu^2)])^{n/2}
\end{equation*}

此时,第$n$阶标准矩有:
\begin{equation*}
    \tilde{\mu}_n = \frac{\mu_n}{\sigma^n} = E\left[ \left(\frac{x-\mu}{\sigma}\right)^n \right]
\end{equation*}

由一阶中心矩为$0$,可知一阶标准矩(first standardized moment)也为$0$。而二阶标准矩(second standardized moment)则有:
\begin{equation*}
    \tilde{\mu}_2 = \frac{\mu_2}{\sigma^2} = \frac{E[(x-\mu)^2]}{\left(E[(x-\mu)^2]\right)^{2/2}} = 1
\end{equation*}

\subsection*{偏度(skewness)}

三阶标准矩(third standardized moment)为\textbf{偏度}:
\begin{equation*}
    \tilde{\mu}_3 = \frac{\mu_3}{\sigma^3} = \frac{E[(x-\mu)^3]}{\left(E[(x-\mu)^2]\right)^{3/2}}
\end{equation*}

偏度分为两种:
\begin{itemize}
    \item 负偏态或左偏态:左侧的尾部更长,分布的主体集中在右侧
    \item 正偏态或右偏态:右侧的尾部更长,分布的主体集中在左侧
\end{itemize}

\subsection*{峰度(kurtosis)}
四阶标准矩(third standardized moment)为\textbf{峰度}:
\begin{equation*}
    \tilde{\mu}_4 = \frac{\mu_4}{\sigma^4} = \frac{E[(x-\mu)^4]}{\left(E[(x-\mu)^2]\right)^{4/2}} 
\end{equation*}

定义\textbf{超值峰度(excess kurtosis)}为峰度$-3$,使得正态分布的峰度为0:
\begin{equation*}
    \text{excess kurtosis} = \tilde{\mu}_4-3
\end{equation*}
\begin{itemize}
    \item 如果超值峰度为正,即峰度值大于3,称为高狭峰(leptokurtic)
    \item 如果超值峰度为负,即峰度值小于3,称为低阔峰(platykurtic)
\end{itemize}

\section{矩母函数}

\subsection{定义}

矩母函数或称为矩生成函数(Moment generating fuction,MGF)或动差生成函数,顾名思义就是产生矩的函数。对于随机变量$X$,其矩生成函数定义为:
\begin{equation*}
    \boxed{
        M_X(t) = \E(e^{tX})
    }
\end{equation*}

离散形式下有:
\begin{equation*}
    \E[e^{tx}] = \sum e^{tx} P(x)
\end{equation*}

而在连续形势下有:
\begin{equation*}
    \E[e^{tx}] = \int_{-\infty}^{\infty} e^{tx} f(x) dx
\end{equation*}

\begin{thm}
    将矩母函数进行n次求导,并令$t=0$则可得到$\E(X^n)$
    \begin{equation*}
        \E(X^n) = \left. \frac{d^n}{dt^n} M_X(t) \right\vert_{t=0}
    \end{equation*}
\end{thm}

\begin{proof}
    对于$e^x$使用泰勒展开有:
    \begin{equation*}
        e^x = 1 + x + \frac{x^2}{2!} + \frac{x^3}{3!} + \dots + \frac{x^n}{n!}
    \end{equation*}

    那么$e^{tx}$的期望为:
    \begin{align*}
        \E[e^{tx}] &= \E\left[1 + tx + \frac{(tx)^2}{2!} + \frac{(tx)^3}{3!} + \dots + \frac{(tx)^n}{n!} \right] \\
        &= \E(1) + t\E(x) + \frac{t^2}{2!}\E(x^2) + \frac{t^3}{3!}\E(x^3) + \dots + \frac{t^n}{n!}\E(x^n) 
    \end{align*}

    对其求一阶导:
    \begin{align*}
        \frac{d}{dt} \E[e^{tx}] 
        &= \frac{d}{dt} \left[ \E(1) + t\E(x) + \frac{t^2}{2!}\E(x^2) + \frac{t^3}{3!}\E(x^3) + \dots + \frac{t^n}{n!}\E(x^n) \right] \\
        &= 0 + \E(x) + t\E(x^2) + \frac{t^2}{2}\E(x^3) + \dots + \frac{t^{n-1}}{(n-1)!}\E(x^n) \\
        & \qquad \text{(代入$t=0$)} \\
        &= 0 + \E(x) + 0 + 0 + \dots + 0 \\
        &= \E(x) 
    \end{align*}
\end{proof}

\subsection{性质}

对于标准正态分布$N\sim(0,1)$的矩母函数,则有:
\begin{align*}
    M_X(t) &= \E (e^{xt}) = \int e^{xt} \frac{1}{\sqrt{\pi}} e^{-\frac{1}{2}x^2}dx \\
    &= \int \frac{1}{\sqrt{\pi}} e^{xt-\frac{1}{2}x^2}dx \\
    &= \int \frac{1}{\sqrt{\pi}} e^{-\frac{1}{2} (x^2 -2xt + t^2 -t^2)}dx \\
    &= \int \frac{1}{\sqrt{\pi}} e^{-\frac{1}{2} (x-t)^2 + \frac{1}{2}t^2}dx \\
    &= e^{\frac{1}{2} t^2} \int \frac{1}{\sqrt{\pi}} e^{-\frac{1}{2} (x-t)^2 }dx \\
    &= e^{\frac{1}{2} t^2}
\end{align*}

对于正态分布$N\sim(\mu,\sigma)$的矩母函数,则有:
\begin{equation*}
    M_X(t) = \E (e^{xt}) = \int e^{xt} \frac{1}{\sigma\sqrt{\pi}} e^{-\frac{1}{2} \left( \frac{x-\mu}{\sigma} \right)} dx
\end{equation*}

此时代换$z=\frac{x-\mu}{\sigma}$,即$x= \sigma z + \mu$,并有$dx=\sigma dz$:
\begin{align*}
    M_X(t) &= \int e^{(\sigma z + \mu)t} \frac{1}{\sigma\sqrt{\pi}} e^{-\frac{1}{2}z^2}dx \\
    &= e^{\mu t} \int e^{\sigma z t} \frac{1}{\sigma\sqrt{\pi}} e^{-\frac{1}{2}z^2}dx \\
    &= e^{\mu t} \int \frac{1}{\sigma\sqrt{\pi}} e^{-\frac{1}{2} (z^2 -2\sigma t z + (\sigma t)^2 -(\sigma t)^2)}dx \\
    &= e^{\mu t} e^{\frac{1}{2} \sigma^2 t^2} \int \frac{1}{\sigma\sqrt{\pi}} e^{-\frac{1}{2} (z - \sigma t)^2}dx \\
    &= e^{\mu t + \frac{1}{2} \sigma^2 t^2}
\end{align*}

\end{document}