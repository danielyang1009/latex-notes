\documentclass[11pt]{article}
\usepackage{amsmath}
% \usepackage{amstext}
\usepackage[a4paper,left=20mm,right=20mm,top=15mm,bottom=15mm]{geometry}
% \usepackage{tikz}
% \usetikzlibrary{trees,matrix}
% \usepackage{cases}
% \usepackage{mathtools}
% \usepackage{tabularx}
% \usepackage{booktabs}
% \usepackage{multirow}
% \usepackage{float}
\usepackage{hyperref}
\usepackage[UTF8]{ctex}
% ctex与amsmath冲突,放最后可解决
\title{Moment}
\author{杨弘毅}
\date{创建: 2021 年 4 月 8 日 \\修改: \today}
\begin{document}
\maketitle

\section{矩的含义}

数学中矩的概念来自物理学,在物理学中,矩表示距离和物理量的乘积。如力与力臂(参考点的距离)的乘积,得到的是力矩(或扭矩)。可以理解为一杆“秤”,“秤”的平衡的两边重量与距离的乘积相同,则能保持平衡。

而在概率论上,可以理解秤为一杆秤的两端的概率为1,中心点概率为0。如一端秤砣重量,为中奖金额$500$元,但中奖概率为千分之一,即离中心点距离为$0.1\%$,那么期望为$0.5$元。可以理解为了使得秤保持平衡,则另一端,在概率为1,其秤砣重量,中奖金额应为$0.5$元。

而这样既可以把期望看成是矩,即距离(概率)乘以力的大小(随机变量):
\begin{equation*}
    E[x] = \sum_i p_i x_i
\end{equation*}


n阶矩可以表示为,其中$f(x)$为概率密度函数(probability density function):
\begin{equation*}
    \mu_n = \int_{-\infty}^{\infty} (x-c)^n f(x) dx
\end{equation*}


\subsection*{Reference}
\url{https://www.zhihu.com/question/19915565/answer/233262673}\\
\indent\url{https://en.wikipedia.org/wiki/Moment_(mathematics)}


\section{原点矩(Raw/crude moment)}

当$c=0$时,称为原点矩。此时则有\textbf{平均数(mean)}或\textbf{期望(expected value)}的连续形式为:
\begin{equation*}
    \mu = E(x) = \int_{-\infty}^{\infty} (x-0)^1 f(x) dx =
    \int_{-\infty}^{\infty} x f(x) dx
\end{equation*}

其离散形式为:
\begin{equation*}
    \mu = E(x) = \sum_i x_i p_i
\end{equation*}

\section{中心矩(Central moment)}

期望值可以成为随机变量的中心,即当$c=E(x)$时
\begin{equation*}
    \mu_n = E[(x-E(x))^n] = \int_{-\infty}^{\infty} (x-E(x))^n f(x) dx
\end{equation*}

同时可知任何变量的一阶中心矩为0:
\begin{align*}
    \mu_1 &= \int_{-\infty}^{\infty} (x-E(x))^1 f(x) dx \\
    &= \int_{-\infty}^{\infty} x f(x) dx - \int_{-\infty}^{\infty} E(x) f(x) dx \\
    &= E(x) - E(x) \int_{-\infty}^{\infty} f(x) dx \\
    &= E(x) - E(x) \times 1 = 0 
\end{align*}

而二阶中心矩(second central moment)为\textbf{方差(Variance)}
\begin{align*}
    \mu_2 &= \int_{-\infty}^{\infty} (x-E(x))^2 f(x) dx \\
    &= \int_{-\infty}^{\infty} x^2 f(x)dx - 2 E(x) \int_{-\infty}^{\infty} x f(x)dx + [E(x)]^2\int_{-\infty}^{\infty}f(x)dx \\
    &= \int_{-\infty}^{\infty} x^2 f(x)dx - 2 E(x) E(x) + [E(x)]^2\times 1 \\
    &= \int_{-\infty}^{\infty} x^2 f(x)dx - [E(x)]^2 \\
    &= E(x^2) - [E(x)]^2 = \sigma^2
\end{align*}

其离散形式则有:
\begin{equation*}
    \text{Var}(x) = \sigma^2 = \sum p_i (x_i - \mu)^2 
\end{equation*}

\subsection*{Reference}
\url{https://en.wikipedia.org/wiki/Central_moment}

\section{标准矩(Standardized moment)}

标准矩为标准化(除以标准差)后的中心矩,第$n$阶中心矩(standardized moment of degree n)有:
\begin{equation*}
    \mu_n = E[(x-\mu)^n] = \int_{-\infty}^{\infty} (x-\mu)^n f(x)dx
\end{equation*}

已知标准差的$n$次方有:
\begin{equation*}
    \sigma^n = \left(\sqrt{E[(x-\mu)^2]}\right)^n = (E[(x-\mu^2)])^{n/2}
\end{equation*}

此时,第$n$阶标准矩有:
\begin{equation*}
    \tilde{\mu}_n = \frac{\mu_n}{\sigma^n} = E\left[ \left(\frac{x-\mu}{\sigma}\right)^n \right]
\end{equation*}

由一阶中心矩为$0$,可知一阶标准矩(first standardized moment)也为$0$。而二阶标准矩(second standardized moment)则有:
\begin{equation*}
    \tilde{\mu}_2 = \frac{\mu_2}{\sigma^2} = \frac{E[(x-\mu)^2]}{\left(E[(x-\mu)^2]\right)^{2/2}} = 1
\end{equation*}

\subsection*{偏度(skewness)}

三阶标准矩(third standardized moment)为\textbf{偏度}:
\begin{equation*}
    \tilde{\mu}_3 = \frac{\mu_3}{\sigma^3} = \frac{E[(x-\mu)^3]}{\left(E[(x-\mu)^2]\right)^{3/2}}
\end{equation*}

偏度分为两种:
\begin{itemize}
    \item 负偏态或左偏态:左侧的尾部更长,分布的主体集中在右侧
    \item 正偏态或右偏态:右侧的尾部更长,分布的主体集中在左侧
\end{itemize}

\subsection*{峰度(kurtosis)}
四阶标准矩(third standardized moment)为\textbf{峰度}:
\begin{equation*}
    \tilde{\mu}_4 = \frac{\mu_4}{\sigma^4} = \frac{E[(x-\mu)^4]}{\left(E[(x-\mu)^2]\right)^{4/2}} 
\end{equation*}

定义\textbf{超值峰度(excess kurtosis)}为峰度$-3$,使得正态分布的峰度为0:
\begin{equation*}
    \text{excess kurtosis} = \tilde{\mu}_4-3
\end{equation*}
\begin{itemize}
    \item 如果超值峰度为正,即峰度值大于3,称为高狭峰(leptokurtic)
    \item 如果超值峰度为负,即峰度值小于3,称为低阔峰(platykurtic)
\end{itemize}

\subsection*{Reference}
\url{https://en.wikipedia.org/wiki/Standardized_moment}

\end{document}