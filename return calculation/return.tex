\documentclass[11pt]{article}
\usepackage{amsmath}
% \usepackage{amstext}
\usepackage[a4paper,left=20mm,right=20mm,top=15mm,bottom=15mm]{geometry}
\usepackage{tikz}
\usetikzlibrary{trees,matrix}
% \usepackage{cases}
% \usepackage{mathtools}
% \usepackage{tabularx}
\usepackage{booktabs}
\usepackage{multirow}
\usepackage{float}
\usepackage[UTF8]{ctex}
% ctex与amsmath冲突,放最后可解决
\title{收益率计算}
\author{杨弘毅}
\date{创建: 2021 年 4 月 2 日 \\修改: \today}
\begin{document}
\maketitle

\section{连续复利收益率(continuously compounded return)}

已知,根据定义自然常数为如下极限值:
\begin{equation*}
    e = \lim_{n \rightarrow \infty} \left(1+ \frac{1}{n}\right)^n
\end{equation*}

假设名义利率(nominal interest rate)为$R$,且一年计息$m$次,则有,一年的实际利率(effective interest rate)为:

\begin{equation*}
    \begin{aligned}
        \text{当$m=1$时} &\quad 1+r \\
        \text{当$m=2$时} &\quad (1+r/2)^2 \\
        \text{当$m=3$时} &\quad (1+r/3)^3 \\
        \dots &\quad \dots \\
        \text{当$m=\infty$时} &\quad \lim_{m \rightarrow \infty} (1+r/m)^m \\
    \end{aligned}
\end{equation*}

将其变形,则有连续复利收益率为:
\begin{equation*}
    \lim_{m \rightarrow \infty} (1+r/m)^m = 
    \lim_{m \rightarrow \infty} (1+\frac{1}{m/r})^{m/r\times r} = e^r
\end{equation*}

例:当名义利率为$12\%$时,有:
\begin{table}[H]
\centering
\begin{tabular}{@{}rll@{}}
\toprule
m(周期)  & 实际利率              & 实际年化利率 \\ \midrule
1(一年)   & $\frac{0.12}{1}=0.12$  & $(1.12)^1-1=0.12$ \\
2(半年)   & $\frac{0.12}{2}=0.06$  & $(1.06)^2-1=0.1236$      \\
4(季度)   & $\frac{0.12}{4}=0.03$  & $(1.03)^4-1=0.125509$       \\
12(月度)  & $\frac{0.12}{12}=0.01$ & $(1.01)^{12}-1=0.126162$ \\
52(周度)  & $\frac{0.12}{52}=0.0023$  & $(1.0023)^{52}-1=0.127341$\\
365(日度) & $\frac{0.12}{365}=0.00033$ & $(1.00033)^365-1=0.127475$ \\
$\infty$ (连续)  & \multicolumn{2}{l}{$ \lim_{m \rightarrow \infty} (1+\frac{0.12}{m})^{m} - 1 = e^{0.12} = 0.127497
$}            \\ \bottomrule
\end{tabular}
\end{table}
\end{document}