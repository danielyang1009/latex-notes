\documentclass[11pt]{article}
\usepackage[a4paper,left=20mm,right=20mm,top=15mm,bottom=15mm]{geometry}
\usepackage{amsmath}
% \usepackage{amstext}
% \usepackage{float}
% \usepackage{booktabs}
% \usepackage{multirow}
% \PassOptionsToPackage{hyphens}{url}
\usepackage[hyphens]{url}
\usepackage{hyperref}
\usepackage[UTF8]{ctex}

\title{统计知识复习}
\author{杨弘毅}
\date{创建: 2020 年 4 月 9 日 \\修改: \today}
\begin{document}
\maketitle

\section{Definition}

\subsection*{Expected Value}
\noindent
\begin{align*}
    E[X] &= \sum_{i=i}^n x_i p_i =x_1 p_1 + x_2 p_2 + \dots + x_n p_n \\
    &= \int x f(x) dx
\end{align*}

\subsection*{Variance}
\noindent
\begin{align*}
\text{Var}(X) &= \text{Cov}(X,X) = \sigma_X^2\\ 
&= E[(X-E[X])^2] \\
&= E[X^2 - 2XE[X] + E[X]^2] \\
&= E[X^2] - 2E[X]^2 + E[X]^2 \\
&= E[X^2] - E[X]^2
\end{align*}

\subsection*{Covariance}
\noindent
\begin{align*}
\text{Cov}(X,Y) &= E[(X-E(X))(Y-E(Y))] \\
&= E[XY - XE[Y] -YE[X]+ E[X]E[Y]] \\
&= E[XY] - E[X]E[Y] - E[X]E[Y] + E[X]E[Y] \\
&= E[XY] - E[X]E[Y]
\end{align*}

\subsection*{Correlation Coefficient}
\noindent
\begin{align*}
\rho_{X,Y} &= \frac{\text{Cov}(X,Y)}{\sigma_X\sigma_Y} \\
&= \frac{E[(X-E[X])(Y-E[Y])]}{\sigma_X\sigma_Y}
\end{align*}

\section{Expected Value Properties}
\noindent
\begin{align*}
    E[X+Y] &= E[X] + E[Y] \\
    E[aX] &= a E[X] \\
    E[XY] &= E[X]E[Y] \quad \text{(X,Y are independent)}
\end{align*}

\section{Variance Properties}
\noindent
\begin{align*}
    \text{Var}(X + a) &= \text{Var}(X) \\
    \text{Var}(aX) &= a^2\text{Var}(X) \\
    \text{Var}(aX \pm bY) &= a^2\text{Var}(X) + b^2\text{Var}(Y) \pm 2ab \text{Cov}(X,Y) \\
    \text{Var}(\sum_{i=1}^{N} X_i) &= \sum_{i,j=1}^{N}\text{Cov}(X_i,X_j) = \sum_{i=1}^{N}\text{Var}(X_i) + \sum_{i \neq j}\text{Cov}(X_i,X_j) \\
    \text{Var}(\sum_{i=1}^{N} a_i X_i) &= \sum_{i,j=1}^{N}a_i a_j\text{Cov}(X_i,X_j) \\
    &= \sum_{i=1}^{N}a_i^2\text{Var}(X_i) + \sum_{i \neq j} a_i a_j \text{Cov}(X_i,X_j) \\
    &= \sum_{i=1}^{N}a_i^2\text{Var}(X_i) + 2\sum_{1 \leq i \leq j \leq N} a_i a_j \text{Cov}(X_i,X_j)
\end{align*}

\subsection*{Reference}
\url{https://mathworld.wolfram.com/Covariance.html}\\
\indent\url{https://en.wikipedia.org/wiki/Covariance}\\
\indent\url{https://ocw.mit.edu/resources/res-6-012-introduction-to-probability-spring-2018/part-i-the-fundamentals/MITRES_6_012S18_L12AS.pdf}

\section{假设检验(Statistical hypothesis testing)}

\textbf{原假设($\text{H}_0$,null hypothesis)},也称为零假设或虚无假设。而与原假设相反的假设称为\textbf{备择假设($\text{H}_a$,althernative hypothesis)}。假设检验的核心为\textbf{反证法}。在数学中,由于不能穷举所有可能性,因此无法通过举例的方式证明一个命题的正确性。但是可以通过举一个反例,来证明命题的错误。在掷骰子的例子中,在每次掷的过程相当于一次举例,假设进行了上万次的实验,即便实验结果均值为3.5,也无法证明总体的均值为3.5,因为无法穷举。

可以理解为原假设为希望拒绝的假设,或反证法中希望推翻的命题。我们先构造一个小概率事件作为原假设($\text{H}_0$),并假设其正确。如样本均值等于某值,两个样本均值是否相等,样本中的不同组直接是否等概率发生,一般使用等式(小概率)作为原假设。如果抽样检验中小概率事件发生,则说明原假设的正确性值得怀疑。如此时假设实验的结果(样本)远大于或小于理论计算结果3.5,即发生了小概率事件,那么就有理由相信举出了一个反例,这时就可以否定原命题(reject the null hypothesis)。而相反,如果原假设认为均值为3.5,在实验的过程中结果大概率不会偏离这个理论值太多,可以认为我们并没办法举出反例。由于不能直接证明原命题为真,只能说”We can not(fail to) reject the null hypothesis“,无法拒绝原命题。

在需要评估总体数据的时候,由于经常无法统计全部数据,需要从总体中抽出一部分样本进行评估。假设掷骰子一个骰子,其期望为3.5,但假设掷骰子了100次,计算均值为3.47,由于总体的理论值和样本呢的实验值可能存在偏差,误差永远存在,无法避免。那么是否可以认为么3.47“等于”3.5?这时候就需要要界定一个\textbf{显著水平($\alpha$,significant level)},相当于设定一个等于的阈值范围。即多小概率的事情发生,是$10\%$还是$5\%$的概率,使我们认为举出了一个反例,值得去怀疑原命题的正确性。当我们知道随机变量的分布时候,根据所进行的检验,我们可以根据计算出的\textbf{统计量(test statistic)},由于分布已知,统计量对应了一个\textbf{p值(p-value)},即小概率(极端)事件发生的概率,因此在图形上表示为统计量向两侧延申的线下区域。如果这个概率足够低,如小于$\alpha=5\%$,那么就有理由拒绝原假设。

用1-显著水平($1-\alpha$),得到值称为\textbf{置信水平(confidence level)}(概率大小)。置信水平越大,对应的置信区间也越大(随机变量范围)。此时有置信水平为$1-\alpha$,假设置信区间为$(a,b)$,那么有$P(a<\text{随机变量}<b)=1-\alpha$。对于双侧检验,有置信水平为$1-\alpha$(概率大小),两侧拒绝域分别为$\alpha/2$。对于单侧检验,则有单侧拒绝域大小为$\alpha$。

\section{Chi-square distribution}

假设有随机变量$X$服从标准正态分布,即有$X \sim N(0,1)$,此时有随机变量$Q_1=X^2$,则有随机变量$Q_1$服从卡方分布($\chi^2\text{-distribution}$),由于此时只有一个随机变量,因此卡方分布自由度(degree of freedom)为1,即$Q_1 \sim \chi^2(1)$。如随机变量$Q_2 = X_1^2 + X_2^2$,且$X_1$与$X_2$同时服从标准正态分布。则此时$Q_2$服从自由度为2的卡方分布,即$Q_2 \sim \chi^2(2)$。


\subsection*{Goodness of fit}

Pearson's chi-squared test
\begin{equation*}
    \chi^2 = \sum_i^n \frac{(O_i - E_i)^2}{E_i}
\end{equation*}

\begin{list}{-}{}
    \item $O_i$ the number of observations of type i
    \item $E_i$ the expected(theoretical) number of type i
\end{list}

\subsection*{Reference}
\url{https://en.wikipedia.org/wiki/Chi-square_distribution}\\
\indent\url{https://www.khanacademy.org/math/statistics-probability/inference-categorical-data-chi-square-tests}

\section{Probability vs Likelihood}

\subsection{Probability}

P( data | distribution ) = area under curve

P( weight between 32g and 34g | mean = 32 and standard deviation = 2.5) = 0.29

P( weight > 34g | mean = 32 and standard deviation = 2.5) = 0.21

\subsection{Likelihood}

L( distribution | data ) = value of the curve (y)

L( mean = 32 and standard deviation = 2.5 | mouse weights 34g ) = 0.12

L( mean = 34 and standard deviation = 2.5 | mouse weights 34g ) = 0.21

在调整了分布的mean之后,likelihood最大,在mean=34 sigma=2.5的正态分布中,抽中一只34g的老鼠的概率最大

\subsection{Maximum likelihood}

测量了数只老鼠的重量,尝试找到其分布,miximizes the likelihood 找到最大化所有观察重量likelihood的分布,找到mean 和standard deviation




\end{document}